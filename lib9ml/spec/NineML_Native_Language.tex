\documentclass[10pt]{article}

\usepackage{amssymb}
\usepackage{amsmath}
\usepackage{graphicx}
\usepackage{cite}
\usepackage{color}

%% Text layout
\topmargin 0.0cm
\oddsidemargin 0.5cm
\evensidemargin 0.5cm
\textwidth 16cm 
\textheight 21cm

%% Bold the 'Figure #' in the caption and separate it with a period
%% Captions will be left justified
\usepackage[labelfont=bf,labelsep=period,justification=raggedright]
{caption}

%% Remove brackets from numbering in List of References
\makeatletter
\renewcommand{\@biblabel}[1]{\quad#1.}
\makeatother

%% Leave date blank
\date{}

\pagestyle{myheadings}

\newcommand{\minitab}[2][l]{\begin{tabular}{#1}#2\end{tabular}}

%% double square brackets for semantic definitions
\newcommand{\lsem}{[\![}
\newcommand{\rsem}{]\!]}


%% end semantic macros

%% END MACROS SECTION


\begin{document}
\begin{flushleft}

{\Large
{\textbf{The NineML Native Interpreted Language}}\\

}
%% Insert Author names, affiliations and corresponding author email.
Ivan Raikov$^{1,2}$
\\
{\textbf{1}} {University of Antwerp}\\
{\textbf{2}} {Okinawa Institute of Science and Technology}\\
\end{flushleft}


\section*{Introduction}
The NineML native interpreted language is a compact and
complete domain-specific interpreted language that specifies an exact
and minimal yet sufficient grammar for populating the NineML object
model, and therefore can serve as a reference implementation.  It is
referred to as native in that, in contrast to other NineML
implementations of the object model which are embedded in existing
programming languages, the NineML native language implements its own
language grammer, and thus avoids the common problem of embedded
languages that the surrogate language can invade the embedded
language.  Herein we describe the structure of this ``NineML native''
language. 

\subsection*{Bindings}
A fundamental aspect of a programming language is how it uses
names to refer to computational objects.  We refer to NineML names as
labels and to computational objects as values, and we say that labels
identify values. 

In the NineML abstraction layer interpreter, values are named
with the binding construct: \begin{center}
{\small{}\begin{verbatim}
binding pi = 3.14159265
\end{verbatim}
}\end{center}


Once a label has been associated with a value, we can refer to the
value by its label, thus implementing the basic property of name
resolution in the NineML interpreter. A binding consists of a label
and a value. Once a label is declared in a binding, its associated
value can be reached by other bindings in the program, according to
the scoping rules of the NineML interpreter, as described in the
following sections. 

In general, values may have complex structures, and it would not
be practical to have to construct these structures each time we want
to use them. Instead, we can use simple labels to refer to compound
values. Compound values in NineML are constructed by building
computational objects of increasing complexity as a sequence of
bindings. 

\subsection*{Expressions}
The value of a binding can be given as a primitive expression, or
as the result of a compound expression. One kind of primitive
expression type is a number. 

Expressions representing numbers may be combined with an
expression representing a primitive procedure (such as \texttt{+}
or \texttt{*}) to form compound expressions. For example: \begin{center}
{\small{}\begin{verbatim}
binding radius = 1
binding area   = pi * radius * radius
\end{verbatim}
}\end{center}


The value of a compound expression is obtained by applying the
operators specified to the arguments that are the values of the
operands. In the example above, the value of the binding area is
obtaining by applying the multiplication operator to the values of pi
and radius. 

\subsection*{Functions and Local Names}
Function definitions in the NineML interpreter allow a compound
operation to be given parameterized as a unit. 

For example, we can declare the following function for squaring
numbers: \begin{center}
{\small{}\begin{verbatim}
binding square = function (x)
                 binding result = x * x
                 return result
\end{verbatim}
}\end{center}


We have here a binding, which has been given the name square. The
value of the binding is a function that represents the operation of
multiplying something by itself. The thing to be multiplied is given a
local name, \texttt{x}. 

The general form of a function definition is: \begin{center}
{\small{}\begin{verbatim}
function ( <formal parameters>) = <binding> ... <binding>  return <value>
\end{verbatim}
}\end{center}


The \texttt{<formal parameters>} are the names used within the
body of the function to refer to the corresponding arguments of the
function. The bindings inside the function contain expressions in
which formal parameters are replaced by the actual arguments to which
the procedure is applied. The return value of the function can refer
to the values of the bindings defined in that function. 

Having defined square, we can now use it: \begin{center}
{\small{}\begin{verbatim}
binding area = function (radius)
               binding result = pi * square (radius)
               return result
\end{verbatim}
}\end{center}


The meaning of the function is independent of the parameter names
used by its author. The consequence is that the parameter names of a
function are local to the body of the function. 

\subsection*{Block Structure and Scoping Rules}
The possibility of associating values with labels and later
retrieving them means that the NineML interpreter must maintain some
sort of memory that keeps track of the label-value pairs. This memory
is called the environment. A NineML program has a global environment
associated with it, and every function has a local environment
associated with it. 

The set of expressions for which a binding defines a label is called
the scope of that label.  For example, \begin{center}
{\small{}\begin{verbatim}
binding a = function (x) =

            binding i = 1
            binding j = 2

            binding b = function (x)

                // scope a+b

                binding k = x+i

                binding c = function (x)
                    // scope a+b+c
                    return x+j

 

                return k + c(k)

            // scope a
            return i+b(j)
\end{verbatim}
}\end{center}


In a function definition, the labels declared as the bindings in that
function have the body of the function as their scope. 

Such nesting of definitions is called lexical scoping: each use of
a label in the NineML interpreter is resolved by looking up the label
in the enclosing block definition, and if not found, in the outer
blocks enclosing the current block. 

\subsection*{Module Language}
TODO

\section*{Semantic Definition}
Every programming language presents its own conceptual view of
computation. This view is reflected by the syntactic structure and
reserved words of the language, which are usually terms such as
procedure, package, pointer, and so on. These terms have their
abstract counterparts, which may be called semantic objects, and which
are what users of the language have in mind when they write programs
in it. And it is these objects, not the syntax, that define the
conceptual world of the language. 

Therefore a definition of the language must be in terms of these
semantic objects. Furthermore, the definition must be in mathematical
notation that is independent of a programming language and follows a
set of rules that allow the consistent definition of every semantic
object. 

\subsection*{Formal Framework}
Operational semantics is a formal expression of the common
intuition that program execution can be understood as a step-by-step
process evolved by the mechanical application of a fixed set of
rules. 

Sometimes the rules describe how the state of some physical machine is
changed by executing an instruction. For example, assembly code
instructions are defined in terms of the effect that they have on the
architectural elements of a computer: registers, stack, memory,
instruction stream, etc. 

But the rules may also describe how language constructs affect the
state of some abstract model for domain-specific computation. 

The key notation of the method takes the form: \begin{equation*}
C\;\vdash{}\;P\;\Rightarrow{}\;M\end{equation*}
 and is read as, ``Against the context $C$, the phrase $P$ evaluates to the meaning $M$''. 

This notation underlies the form of operational semantics used in
this document, which is called big-step operational semantics, and is
also known as natural semantics. 

\subsection*{Semantic Rules}
A semantic rule has the form \begin{equation*}
\frac{antecedents}%
{consequent}\end{equation*}
 where the antecedents and the consequent contain formulas with
variables (described next). Informally, the rule asserts: ``If the
transitions specified by the antecedents are valid, then the
transition specified by the consequent is valid.''  

TODO: Axioms

Formulas are divided in two kinds: sequents and conditions.
Conditions are restriction on the applicability of the rule: a
variable may not occur free somewhere, a value must satisfy some
predicate, some relation must hold between two variables.  

A sequent has two parts, an antecedent (on the left) and a
consequent (on the right), and we use the turnstile symbol $\vdash{}$ to separate these parts. It takes the form: \begin{equation*}
s_{1}\;\vdash{}\;E\;\Rightarrow{}\;s_{2}\end{equation*}
 which means that given program state $s_{1}$ the expression $E$ evaluates to program state $s_{2}$. 

\subsection*{Syntax of the Core Language}
\begin{tabular*}{5cm}{l@{\extracolsep{1cm}}l}
$Expression$$\:::=\:$$L$ &[Literal (symbolic or numeric constant)]\\ $\qquad\;\bigm\vert\;$
$I$ &[Identifier]\\ $\qquad\;\bigm\vert\;$
$\langle{}E_{1}\;\texttt{,}\;E_{2}\rangle{}$ &[Tuple constructor]\\ $\qquad\;\bigm\vert\;$
$\texttt{fn}\quad{}I_{parameter}\quad{}E_{body}$ &[Function constructor]\\ $\qquad\;\bigm\vert\;$
$\texttt{if}\quad{}E_{1}\quad{}E_{2}\quad{}E_{3}$ &[Conditional]\\ $\qquad\;\bigm\vert\;$
$E_{proc}\quad{}E_{arg}$ &[Function application]\\ $\qquad\;\bigm\vert\;$
$\texttt{let}\quad{}I_{lb}\texttt{=}E_{lb}\quad{}\texttt{in}\quad{}E_{body}$ &[Expression with local binding]\\ $\qquad\;\bigm\vert\;$
$\texttt{binding}\quad{}I_{val}\quad{}\texttt{=}\quad{}E_{val}$ &[Creates a new binding in the host namespace]\\ ~&~\\
$Literal$$\:::=\:$$N$ &[Numeric Literal]\\ $\qquad\;\bigm\vert\;$
$I_{symbol}$ &[Symbolic Literal]\\ \end{tabular*}
\subsection*{Semantics for the Core Language}
In the Core Language, the evaluation of an  expression always
yields a value, and all sequents take the form:  $\rho{}\;\vdash{}\;E\;\Rightarrow{}\;\alpha{}$ where $E$ is an expression, $\rho{}$ is an environment, and $\alpha{}$ is the result of the evaluation of $E$
in $\rho{}$. 

\subsubsection*{Semantic Values and Environments}
Values in the core language can be one of the following: \begin{enumerate}
\item integers;
\item IEEE 754 floating-point numbers;
\item boolean values $true$ and $false$
\item closures of the form $\langle{}\lambda{}P.E\rho{}\rangle{}$, where $E$ is an expression and $rho$ is an environment; in other words, a closure is a pair denoting a function and its execution environment; 
\item predefined primitive functions;
\item tuples of semantic values of the form $\langle{}alpha\;beta\rangle{}$, which may in turn be tuples, so that trees of semantic values can be constructed. 
\end{enumerate}


An environment $\rho{}$ is an ordered list of pairs ${\mathrm\mathbf{}(}P\;\;\;\rightarrow{}\;\;\;\alpha{}{\mathrm\mathbf{})}$ where $P$ is pattern and a $\alpha{}$ a value. The initial
environment associates primitive functions to a few predefined
operators, such as \texttt{+}, \texttt{-} etc. 

\subsubsection*{Semantic Rules}
Axioms 1 to 4 state that literals are in normal form, i.e. they immediately yield semantic values: 

\begin{equation}
\rho{}\;\vdash{}\;n\;\Rightarrow{}\;N\end{equation}
\begin{equation}
\rho{}\;\vdash{}\;r\;\Rightarrow{}\;R\end{equation}
\begin{equation}
\rho{}\;\vdash{}\;\texttt{true}\;\Rightarrow{}\;true\end{equation}
\begin{equation}
\rho{}\;\vdash{}\;\texttt{false}\;\Rightarrow{}\;false\end{equation}
Axiom 5 asserts that an expression of the form $\lambda{}P.E$ is also in normal form. The
value obtained is a closure, pairing this expression with the
environment. 

\begin{equation}
\rho{}\;\vdash{}\;\lambda{}P.E\;\Rightarrow{}\;\langle{}\lambda{}P.E\rho{}\rangle{}\end{equation}
Axiom 6 states that the value associated to an identifier is to be looked up in the environment $\rho{}$. 

\begin{equation}
\rho{}\;\vdash{}\;I\;\Rightarrow{}\;\texttt{ident-find}(I,\:\rho{})\end{equation}
Evaluation of conditional expressions is specified by the next two
rules, 7 and 8. The rules show that evaluation of $E_{1}$, $E_{2}$ and $E_{3}$ takes place in the same
environment, and has no side effects. 

\begin{equation}
\frac{{\mathrm\mathbf{}(}\rho{}\;\vdash{}\;E_{1}\;\Rightarrow{}\;true{\mathrm\mathbf{})}\;{\mathrm\mathbf{}(}\rho{}\;\vdash{}\;E_{2}\;\Rightarrow{}\;\alpha{}{\mathrm\mathbf{})}}%
{\rho{}\;\vdash{}\;\texttt{if}\;E_{1}\;\texttt{then}\;E_{2}\;\texttt{else}\;E_{3}\;\Rightarrow{}\;\alpha{}}\end{equation}
\begin{equation}
\frac{{\mathrm\mathbf{}(}\rho{}\;\vdash{}\;E_{1}\;\Rightarrow{}\;false{\mathrm\mathbf{})}\;{\mathrm\mathbf{}(}\rho{}\;\vdash{}\;E_{3}\;\Rightarrow{}\;\alpha{}{\mathrm\mathbf{})}}%
{\rho{}\;\vdash{}\;\texttt{if}\;E_{1}\;\texttt{then}\;E_{2}\;\texttt{else}\;E_{3}\;\Rightarrow{}\;\alpha{}}\end{equation}
Rule 9 is the introduction rule for tuples of values. Both components of the tuple must be evaluated. 

\begin{equation}
\frac{{\mathrm\mathbf{}(}\rho{}\;\vdash{}\;E_{1}\;\Rightarrow{}\;\alpha{}{\mathrm\mathbf{})}\;{\mathrm\mathbf{}(}\rho{}\;\vdash{}\;E_{2}\;\Rightarrow{}\;\beta{}{\mathrm\mathbf{})}}%
{\rho{}\;\vdash{}\;E_{1}\;\texttt{,}\;E_{2}\;\Rightarrow{}\;{\mathrm\mathbf{}(}\langle{}\alpha{}\;\beta{}\rangle{}{\mathrm\mathbf{})}}\end{equation}
The next rule deals with function application. Type-checking must
ensure that the operator of an application can only evaluate to a
functional value, i.e. a closure. This closure is taken apart. The
closure's body is evaluated in the closure's environment, to which a
parameter association has been added. Since $E_{2}$ is
evaluated, the Core Language uses call-by-value. 

\begin{equation}
\frac{{\mathrm\mathbf{}(}\rho{}\;\vdash{}\;E_{1}\;\Rightarrow{}\;\langle{}\lambda{}P.E\rho{}_{1}\rangle{}{\mathrm\mathbf{})}\;{\mathrm\mathbf{}(}\rho{}\;\vdash{}\;E_{2}\;\Rightarrow{}\;\alpha{}{\mathrm\mathbf{})}\;{\mathrm\mathbf{}(}\rho{}_{1}\cdot{}P\mapsto{}\alpha{}\;\vdash{}\;E\;\Rightarrow{}\;\beta{}{\mathrm\mathbf{})}}%
{\rho{}\;\vdash{}\;E_{1}\;E_{2}\;\Rightarrow{}\;\beta{}}\end{equation}
Rule 11 deals with local bindings. 

\begin{equation}
\frac{{\mathrm\mathbf{}(}\rho{}\;\vdash{}\;E_{2}\;\Rightarrow{}\;\alpha{}{\mathrm\mathbf{})}\;{\mathrm\mathbf{}(}\rho{}\cdot{}P\mapsto{}\alpha{}\;\vdash{}\;E_{1}\;\Rightarrow{}\;\beta{}{\mathrm\mathbf{})}}%
{\rho{}\;\vdash{}\;\texttt{let}\;P\;=\;E_{2}\;\texttt{in}\;E_{1}\;\Rightarrow{}\;\beta{}}\end{equation}
\subsection*{Semantics for the Module Language}
TODO

\end{document}
