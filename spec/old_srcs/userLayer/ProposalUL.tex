\documentclass{article}
\usepackage[margin=1in]{geometry}
\usepackage{hyperref}
\usepackage{graphicx}
\usepackage{color}

\definecolor{issuecolor}{rgb}{0.8,0.8,0.8}
\newcommand{\issue}[1]{%
\begin{center}
\colorbox{issuecolor}{\parbox{0.8\linewidth}{\textbf{Issue:} #1}}
\end{center}%
}

\newcommand{\suggestion}[2]{%
\begin{center}
\colorbox{issuecolor}{\parbox{0.8\linewidth}{\textbf{#1:} #2}}
\end{center}%
}


\title{INCF Task Force for a Standard Language\\in Neural Network Models\\
\vskip 0.35in
\Large{\bf NINeML (9ML) version 0.91 General Description}
\vskip 0.15in
\Large{Proposal v.10 ID:AG-YLF-20110116.01}}
\author{Maintained by Anatoli Gorchetchnikov}

\begin{document}
\maketitle

\abstract
This draft intends to describe our effort to develop a standardized language
for neuronal network modeling using a set of reference models. The aim of this
first implementation is to design a benchmark for the language using a simple
model and then adapt the language to more complex networks.

This work currently summarizes the different ideas and propositions elaborated
during six Task Force meetings through an empirical approach and will serve
as a basis for further work.

\paragraph{Changes from version 9:}
\begin{itemize}
\item Replaced ``node'' with ``component'' as was decided at the Antwerp 2010 meeting.
\item Replaced references to ``core semantics'' with references to ``abstraction layer''
because the term ``core semantics'' is no longer used.
\item Removed the appendix with example models. Please refer to the
\href{http://www.incf.org/core/programs/modeling/wiki/documentation/user-layer}
{\underline{User Layer wiki section}} instead.
\item As Antwerp 2010 decided, properties are represented by a generic tag and specified
by attribute {\tt name}.
\item Added the reference to units of measurements dimensionality checking against
the abstraction layer as was decided in Antwerp 2010.
\item Added syntax for non-numeric properties (Boolean and enumerated).
\item Kept the specific tags for properties that are references to other components,
e.g. space in region component or random distribution in parameter values.
\end{itemize}

\paragraph{Changes from version 8:}
\begin{itemize}
\item added footnotes for some issues raised in Stockholm June meeting
\item cleaned the {\tt definition} tag syntax according to Stockholm
decisions
\item added brief description of interactions between user layer and
simulation software that does not support abstraction layer
\item removed random number generator components as decided in Stockholm
\item added more examples of random distributions in 1, 2 and 3D
\item added random projection when a probability of projection between any
cell in source and any cell in target is given
\item added space, region, and layout components as a first draft of geometrical
concepts (based on NetworkML proposal by Padraig and Robert as well as
Stockholm discussions)
\item removed recurrent projections that can be described on the population
level
\item cut out the previous approach to cell positioning within population
\end{itemize}

\tableofcontents

\section{Introduction}

With the increase of the Computational Neuroscience community and the number of
work related to neural network modeling, the INCF identified a need for standards
and guidelines, which will allow optimized model sharing, insure model reproducibility
throughout the different software and is crucial for the scientific good practice.
To create such a standard for neuronal network modeling, the INCF created a program
aiming at developing a common standardized descriptive language for neural models.

The development of such a language is the first mandate of this Task Force. This
language is developed using the modern descriptive technology including XML and
possibly UML and should allow an unambiguous definition of the model of network both
conceptually and mathematically and insure complete software independence.
In addition this language is designed to be self-consistent and highly
flexible, allowing addition of new models and mathematical descriptions without
modification of the previous structure and organization of the language. This
standard is a two layered language: a user description layer which represents
the conceptual description of the model in biological terms and an abstraction layer
that will define the mathematics behind each of the concepts described at the user
layer level.

User description layer is intended to be primarily machine readable. XML syntax
defined in this document is designed with considerations of ease of parsing,
verification, and automatic model construction in mind. This does not prevent the
advanced user from editing the user layer XML descriptions by hand, but the
primary means for creation of these descriptions are the software tools that
will convert the GUI-based or script-based representations of objects and
properties into a valid XML code.

The roles of two layers are delineated clearly by implementing an object oriented
type of coding, the mathematical formalism being described in the abstraction layer
and referenced in the user layer. This approach helps to complete
the self-consistency requirement as any term used in the user layer is automatically
linked to its formal definition in the abstraction layer. In this manuscript we focus
on the organization of the user layer of the language, provide the syntax of linking
with the abstraction layer, and describe in mathematical terms what the corresponding
abstraction layer objects are.

In order to achieve the ultimate goals of the Task Force, meaning creating a first
version of the language and propose it to the community of modelers, we decided
to constrain our approach by selecting few seminal papers to design a working
version of the language. The first specification presented here is based on
several models selected by the Task Force and aim to be further tested by extending
the network repertoire to more complex networks. This insures that the language
presents a logical organization, which makes it robust to new additions and further
modifications.

This manuscript so far only covers modifications of four simple forms of spiking
neurons: Poisson spiker, two parametrizations of the leaky integrate and fire neuron
and an Izhikevich neuron. Three types of postsynaptic responses: the delta synapse, the
alpha synapse and the exponential synapse are covered. One plasticity rule (power
law STDP) and a case of non-plastic weights, two currents: injected and
conductance-based, and one conductance (h-conductance) are also covered.
In addition to that this document covers six random distributions, four
connectivity patterns, five coordinate systems, five regions of space and seven
layouts. Throughout the process of verification of this
first implementation we will then expand the number of models to all the common
forms of spiking neurons and synapses, widely used in the literature, as well as
different types of network architectures.

In order to achieve this purpose, we agreed on the need of ontology of the neural
network, requiring the compilation of all or most of the publication on neural
network modeling. This work currently in progress will also lead to a better
definition of terminology the standard should use. We agreed that the use of
specific names such as `layer' or `neuron' is confusing and we chose to use neutral
terms such as `component', `set' or `group'. Further refinement of these terms can
be based on the current state-of-art of the neuronal network modeling. In the same
time, the current version of the language does
not cover much of the initialization and run-time specification. NINeML is
designed to provide network description rather than simulation description,
although there are cases when initialization of certain variables is indeed
part of the model rather than of a simulation setup. Drawing the line between
these two will happen during the development process using an extended library
of neural networks.

The first part of this document presents descriptions of single components
(cells, synapses, inputs) with examples of user layer code, abstraction layer
behind this code, and discussion of important issues. The second part deals with the
grouping of these individual components while the third part addresses the
proposition for the connectivity description. For these last two parts the abstraction
layer should deal with the geometrical representation of cell arrangement as well
as coordinate systems in the feature space. These two parts of abstraction layer
are not currently fully implemented and thus are omitted from the current version
of the manuscript.

\href{http://www.incf.org/core/programs/modeling/wiki/documentation/user-layer}
{\underline{User Layer wiki section}} includes a number of examples of user layer
descriptions.

Formal grammar definition shall be provided in the appendix.

\section{Components and Properties}

The basic building block of NINeML is called component. In the user layer description
component is a reference to an object defined in the abstraction layer.
Abstraction layer defines the mathematics of the component and is referred
in the user layer by {\tt definition} tag with language attribute set to ``NINeML'':
\begin{verbatim}
<definition language="NINeML">
  http://www.NINeML.org/neurons/IaF_tau.xml
</definition>
\end{verbatim}

For the ease of extensibility of the language the standard
allows to put references to user space definitions. For example user defines some
neuron through syntax used by abstraction layer in a separate file and then refers
to it in the {\tt definition} similar to how references to NINeML database definitions
are made:
\begin{verbatim}
<definition language="NINeML">
  file://home/someuser/neurons/My_IaF.xml
</definition>
\end{verbatim}

Also for the purpose of extensibility, the initial version of the standard allows
external object definitions. Since these are completely outside of NINeML and
implementation dependent, the {\tt definition} tag has an attribute {\tt language}
that has to be recognized by the parser of a specific simulator. No further
assumptions about the contents of such definitions are made.
\begin{verbatim}
<definition language="ndf">
  ...
</definition>
\end{verbatim}
The procedure shall be established to ensure that these external definitions
are translated into NINeML definitions that can be used by multiple simulators
as the time progresses.

Each of the components has a user-given ID or {\tt name} and a set of properties
that is defined in the above definition and instantiated in the user layer.
We decided that properties should have generic tags for extensibility purpose
and further described by {\tt name} attribute. Since these properties are unique
for every definition, the names that are used for them are predefined by NINeML
for abstraction layer definitions and by 3rd party for external
definitions.

\issue{This contradicts our usual policy in making names user-defined,
maybe this attribute should be called differently.}

The abstraction layer specification or the documentation for the external definition
provides the mapping between mathematical description of the object and the
corresponding properties. Here some of these mappings for NINeML abstraction
layer definitions are also provided.

User can add short notes to each component description. These notes are
intended to provide a specific reference to the research paper page
where the component is described and similar kind of information. These
notes are not intended to duplicate the abstraction layer documentation
or any other documentation, thus they shall not provide mathematical
description and other details of the component implementation. Note can
contain text or link to an Internet resource.

As a result, a complete description of a component in the user layer consists of
three parts: reference to a definition, a list of properties, and an optional
note:
\begin{verbatim}
<component name="E_ext">
  <definition language="NINeML">http://www.NINeML.org/neurons/Poisson.xml</definition>
  <properties>
    <quantity name="rate">...</quantity>
  </properties>
  <note><text>Poisson input (page 185)</text></note>
</component>
\end{verbatim}
The page number in the note above as well as in other note examples
in this document refer to \cite{brunel-00} unless stated otherwise.

\suggestion{Hugo}{Use identical syntax for name and for other properties,
does not look good with current syntax, e.g:}
\begin{verbatim}
<component>
  <definition language="NINeML">http://www.NINeML.org/neurons/Poisson.xml</definition>
  <properties>
    <string name="name">E_ext</string>
    <quantity name="rate">...</quantity>
  </properties>
  <note><text>Poisson input (page 185)</text></note>
</component>
\end{verbatim}

To reduce the size of the resulting network description user can refer to
already described component by user-given name instead of providing link to
abstraction layer. In this case the properties of the component that have to be
redefined are stated explicitly, the properties that are inherited from the
original description are omitted:
\begin{verbatim}
<component name="E_ext2">
  <reference>E_ext</reference>
  <properties>
    <quantity name="rate">...</quantity>
  </properties>
</component>
\end{verbatim}
In this case the component follows the same description as the one shown in the
previous listings except for different {\tt rate}. Note that the reference
is done by the user-defined ID ({\tt name}).

If the simulator only supports user layer, then the simulator developers
can create mappings directly between the reference to a definition in the user
layer description of the component and the intrinsic simulator code that implements
the same mathematics.

\subsection{Property Description}
\label{secProp}

NINeML uses generic tags for properties combined with the {\tt name} attribute. These
attributes are provided by the abstraction layer or external definition of the parent
component. The user can set the value of the property as well as the units of
measurements. The {\tt name} is checked against the abstraction layer for existence,
and the units are checked whether they match the dimensionality of the corresponding
property in the abstraction layer definition. User can add a note to a property in a
way similar to adding notes to components described above.

There are two kinds of properties: values of the first kind are given to the
model by the user and stay fixed, values of a second kind are computed within
the model during simulation. For all practical reasons the syntax of the user
layer descriptions is identical for both kinds of properties. For example
\begin{verbatim}
<quantity name="rate">
  <value>
    <scalar>10</scalar>
    <unit>unit per s</unit>
  </value>
</quantity>
\end{verbatim}
sets a parameter value to a fixed number, while
\begin{verbatim}
<quantity name="voltage">
  <value>
    <scalar>-60</scalar>
    <unit>mV</unit>
  </value>
</quantity>
\end{verbatim}
sets the initial value of the variable.

\issue{Syntax shall be defined to provide compound descriptions of values, for
example when value is computed from some other values. An example of a simple
syntax requiring math parsing:}
\begin{verbatim}
<quantity name="weight">
  <value>
    <reference>reference.properties.quantity[@name="weight"].value*(-1)</reference>
  </value>
</quantity>
\end{verbatim}
Note that the tag {\tt reference} above can be replaced with something specifically
requesting the parsing of math expression and calculating the value:
\begin{verbatim}
<quantity name="weight">
  <value>
    <calculate>reference.properties.quantity[@name="weight"].value*(-1)</calculate>
  </value>
</quantity>
\end{verbatim}

As NINeML does not provide any default values for properties, it is a job of the
user to provide initial values for all defined properties. To ensure the integrity
of the model NINeML requires all initial values to be set in the user layer
description. For batch simulations and other modifications any of the values
given in the user layer can be overwritten by a simulation setup description,
but this is outside of the scope of the current version of NINeML.

In addition to numerical properties for which the generic tag {\tt quantity}
is appropriate, there are other properties that have different types. An example
above used tag {\tt string} for the name property if we decide to make it into
a property. While working with the different models there were two more property
types that popped out.

The first is enumerated type, for example the random
connectivity can be calculated as convergent (100 incoming random connections
to a target neuron), divergent (100 outgoing random connections from a source
neuron), or fixed probability (for every source-target pair there is a fixed
chance of connection). Syntax can be the following:
\begin{verbatim}
<selection name="direction">convergent</selection>
\end{verbatim}

The second type is Boolean. Again, in connectivity description important
Boolean properties are whether self-projections are allowed and whether
multiple connections between same two neurons are allowed. Syntax can be the
following:
\begin{verbatim}
<boolean name="self-projection">True</boolean>
\end{verbatim}

Properties described above have fixed values for all instances of the
respective component and are determined at description level. In addition to that
some properties may vary between different instances based on some random
distribution, spatial coordinate, geometry, and other values that are
determined at the time each instance is created. These can be termed
instantiation level properties and are discussed in the following
subsections.

\subsubsection{Properties with Values Drawn from Random Distributions}
\label{randProp}

These are the simplest out of instantiation level properties because
they do not depend on some other properties outside of their parent component.
A reference to a random distribution which is the same for all instances
of the component is sufficient to set the value for this type of properties:
\begin{verbatim}
<quantity name="voltage">
  <value>
    <random><reference>uniform_55-65</reference></random>
    <unit>mV</unit>
  </value>
</quantity>
\end{verbatim}
Here the property value is set randomly according to a definition defined by a
reference to an object described elsewhere in the network that defines a uniform
distribution as discussed in section \ref{randDist}.

\subsubsection{Properties with Values Dependent on Geometry of the Network}

In the simplest case the property of this type refers to a spatial coordinate
of the particular instance and calculates its value from this coordinate.
Here the reference to the coordinate system component shall be given. This reference
should match one of the coordinate systems that this instance is positioned on.
Also some function has to be described that allows the transformation of one, two,
or three dimensional coordinate value into a single property value. This
function can be defined using an inline abstraction layer definition, reference
to abstraction layer database or MathML definition.

More complex cases can stem from here if coordinate itself is insufficient to
define the property, for example the distance between this instance and some other
instance possibly in another population determines the property value. NINeML
shall insure that both populations are laid out on the same coordinate systems,
reference to the system used shall be provided as above (but is necessary only
if multiple systems used for population).

These are preliminary thoughts, until specific examples will be derived no syntax
can be determined.

\subsection{Random Distribution Components}
\label{randDist}

This set of components allows to define random distributions with corresponding
parameters.

\subsubsection{1D Uniform Distribution}
\label{uniform1D}

Provides 1D uniform distribution.

\paragraph{General Definition}
This distribution generates random numbers uniformly distributed between
{\tt lowerBound} and {\tt upperBound}. This number does not need to have
units of measurements attached as they will be attached to it at the
point where the distribution is used as shown in section~\ref{randProp}.

\paragraph{User Layer Description}
\begin{verbatim}
<component name="uniform_-55-65mV">
  <definition language="NINeML">
    http://www.NINeML.org/distributions/Uniform1D.xml
  </definition>
  <properties>
    <quantity name="lowerBound">
      <value>
        <scalar>-65</scalar>
        <unit>none</unit>
      </value>
    </quantity>
    <quantity name="upperBound">
      <value>
        <scalar>-55</scalar>
        <unit>none</unit>
      </value>
    </quantity>
  </properties>
</component>
\end{verbatim}

\subsubsection{2D Uniform Distribution}
\label{uniform2D}

Provides 2D uniform distribution.

\paragraph{General Definition}
This distribution generates random numbers uniformly distributed within a 2D
rectangle defined by {\tt lowerBoundX}, {\tt lowerBoundY} and {\tt upperBoundX},
{\tt upperBoundY}.

\paragraph{User Layer Description}
\begin{verbatim}
<component name="2D uniform">
  <definition language="NINeML">http://www.NINeML.org/distributions/Uniform2D.xml</definition>
  <properties>
    <quantity name="lowerBoundX">
      <value>
        <scalar>-65</scalar>
        <unit>none</unit>
      </value>
    </quantity>
    <quantity name="upperBoundX">
      <value>
        <scalar>-55</scalar>
        <unit>none</unit>
      </value>
    </quantity>
    <quantity name="lowerBoundY">
      <value>
        <scalar>-65</scalar>
        <unit>none</unit>
      </value>
    </quantity>
    <quantity name="upperBoundY">
      <value>
        <scalar>-55</scalar>
        <unit>none</unit>
      </value>
    </quantity>
  </properties>
</component>
\end{verbatim}

\subsubsection{3D Uniform Distribution}
\label{uniform3D}

Provides 3D uniform distribution.

\paragraph{General Definition}
This distribution generates random numbers uniformly distributed within a 3D
parallelepiped defined by {\tt lowerBoundX}, {\tt lowerBoundY}, {\tt lowerBoundZ}
and {\tt upperBoundX}, {\tt upperBoundY}, {\tt upperBoundZ}.

\paragraph{User Layer Description}
\begin{verbatim}
<component name="2D uniform">
  <definition language="NINeML">http://www.NINeML.org/distributions/Uniform3D.xml</definition>
  <properties>
    <quantity name="lowerBoundX">
      <value>
        <scalar>-65</scalar>
        <unit>none</unit>
      </value>
    </quantity>
    <quantity name="upperBoundX">
      <value>
        <scalar>-55</scalar>
        <unit>none</unit>
      </value>
    </quantity>
    <quantity name="lowerBoundY">
      <value>
        <scalar>-65</scalar>
        <unit>none</unit>
      </value>
    </quantity>
    <quantity name="upperBoundY">
      <value>
        <scalar>-55</scalar>
        <unit>none</unit>
      </value>
    </quantity>
    <quantity name="lowerBoundZ">
      <value>
        <scalar>-65</scalar>
        <unit>none</unit>
      </value>
    </quantity>
    <quantity name="upperBoundZ">
      <value>
        <scalar>-55</scalar>
        <unit>none</unit>
      </value>
    </quantity>
  </properties>
</component>
\end{verbatim}

\subsubsection{1D Normal Distribution}
\label{normal1D}

Provides 1D normal (Gaussian) distribution.

\paragraph{General Definition}
This distribution generates random numbers according to a 1D normal distribution
with mean {\tt mean} and standard deviation {\tt stdDev}.

\paragraph{User Layer Description}
\begin{verbatim}
<component name="normal_1.3-0.25">
  <definition language="NINeML">http://www.NINeML.org/distributions/Normal1D.xml</definition>
  <properties>
    <quantity name="mean">
      <value>
        <scalar>1.3</scalar>
        <unit>none</unit>
      </value>
    </quantity>
    <quantity name="stdDev">
      <value>
        <scalar>0.25</scalar>
        <unit>none</unit>
      </value>
    </quantity>
  </properties>
</component>
\end{verbatim}

\subsubsection{2D Normal Distribution}
\label{normal1D}

Provides 2D normal (Gaussian) distribution.

\paragraph{General Definition}
This distribution generates random numbers according to a 2D normal distribution
with mean {\tt meanX}, {\tt meanY} and standard deviations {\tt stdDevX}, {\tt stdDevY}.

\paragraph{User Layer Description}
\begin{verbatim}
<component name="2D normal">
  <definition language="NINeML">http://www.NINeML.org/distributions/Normal2D.xml</definition>
  <properties>
    <quantity name="meanX">
      <value>
        <scalar>1.3</scalar>
        <unit>none</unit>
      </value>
    </quantity>
    <quantity name="stdDevX">
      <value>
        <scalar>0.25</scalar>
        <unit>none</unit>
      </value>
    </quantity>
    <quantity name="meanY">
      <value>
        <scalar>1.3</scalar>
        <unit>none</unit>
      </value>
    </quantity>
    <quantity name="stdDevY">
      <value>
        <scalar>0.25</scalar>
        <unit>none</unit>
      </value>
    </quantity>
  </properties>
</component>
\end{verbatim}

\subsubsection{3D Normal Distribution}
\label{normal1D}

Provides 3D normal (Gaussian) distribution.

\paragraph{General Definition}
This distribution generates random numbers according to a 3D normal distribution
with mean {\tt meanX}, {\tt meanY}, {\tt meanZ} and standard deviations {\tt stdDevX},
{\tt stdDevY}, {\tt stdDevZ}.

\paragraph{User Layer Description}
\begin{verbatim}
<component name="normal_1.3-0.25">
  <definition language="NINeML">http://www.NINeML.org/distributions/Normal3D.xml</definition>
  <properties>
    <quantity name="meanX">
      <value>
        <scalar>1.3</scalar>
        <unit>none</unit>
      </value>
    </quantity>
    <quantity name="stdDevX">
      <value>
        <scalar>0.25</scalar>
        <unit>none</unit>
      </value>
    </quantity>
    <quantity name="meanY">
      <value>
        <scalar>1.3</scalar>
        <unit>none</unit>
      </value>
    </quantity>
    <quantity name="stdDevY">
      <value>
        <scalar>0.25</scalar>
        <unit>none</unit>
      </value>
    </quantity>
    <quantity name="meanZ">
      <value>
        <scalar>1.3</scalar>
        <unit>none</unit>
      </value>
    </quantity>
    <quantity name="stdDevZ">
      <value>
        <scalar>0.25</scalar>
        <unit>none</unit>
      </value>
    </quantity>
  </properties>
</component>
\end{verbatim}

\subsubsection{More subtypes here}

\subsection{Conductance Components}
\label{cond}

Conductance components describe the dynamics of channel conductances. So far we ran
into one description of such conductance in Hill and Tononi model (\cite{hill-tononi-05}).

\subsubsection{h Conductance}
\label{h-cond}

\paragraph{General Definition}

\paragraph{User Layer Description}
\begin{verbatim}
<component name="h conductance">
  <definition language="NineML">
    http://www.NineML.org/1.0/conductances/h-conductance.9ml
  </definition>
  <properties>
    <quantity name="V_theta">
      <value>
        <scalar>-75.0</scalar>
        <unit>mV</unit>
      </value>
    </quantity>
      ...
    <quantity name="voltage">
      <value>external</value>
    </quantity>
  </properties>
</component>
\end{verbatim}

Note how the value of voltage at the end is set to external. This is the mechanism
to control this conductance by membrane potential of the neuron it will be inserted
in later. At the point of insertion the following construct can be used:
\begin{verbatim}
<link>
  <from>parentEntity.properties.voltage.value</from>
  <to>reference.conductance.quantity[@name="voltage"].value</to>
</link>
\end{verbatim}

\subsection{Current Components}
\label{currents}

Current components describe the dynamics of currents that affect neuronal
behavior. Note that these currents do not include synaptic currents that
are grouped separately under the post-synaptic response components (section
\ref{secSynapse}).

\issue{Some post-synaptic responses are currents, others are not. Currently this
document describes post-synaptic responses together independently of their nature.
Shall it stay this way or shall post-synaptic currents be moved here and other
post-synaptic responses left in a separate section?}

\subsubsection{Injected Current}
\label{inject}

Currents that are injected in the cell by experimenters during some
protocols.

\paragraph{General Definition}
This current does not have internal dynamics and has a single property
{\tt magnitude}. How to manipulate this current externally (input protocol)
is under discussion. It appears that onset and offset times are important
for reproducibility of the results, but on the other hand they belong more
to the simulation description rather then model description.

\paragraph{User Layer Description}
Here we include onset and duration for the current, although as mentioned above
these belong more in the simulation description. Either way they can use the
same syntax
\begin{verbatim}
<component  name="Injection current">
  <definition language="NineML">
    http://www.NineML.org/currents/injection.9ml
  </definition>
  <properties>
    <quantity name="magnitude">
      <value>
        <scalar>-20</scalar>
        <unit>uA</unit>
      </value>
    </quantity>
    <quantity name="onset">
      <value>
        <scalar>2</scalar>
        <unit>ms</unit>
      </value>
    </quantity>
    <quantity name="duration">
      <value>
        <scalar>3</scalar>
        <unit>ms</unit>
      </value>
    </quantity>
  </properties>
</component>
\end{verbatim}

\subsubsection{Conductance Based Current}
\label{cond-curr}
Conductance-based current is the most common current derived from the equivalent
circuit.

\paragraph{General Definition}

Conductance-based current follows the equation
\begin{equation}
I=(E_r-V_m)\bar g g(t,V_m,...)
\label{cond-current-eq}
\end{equation}
Here the $E_r$ is a reverse potential, $V_m$ is a membrane potential, $\bar g$ is the
maximal conductance and $g(t,V_m,...)$ is a conductance change function on the range of
[0,1] that is defined through conductance component as discussed in section~\ref{cond}.
Correspondence between properties and user layer names:
\begin{itemize}
\item[$V_m$] {\tt voltage}
\item[$\bar g$] {\tt peakConductance}
\item[$E_r$] {\tt reversePotential}
\item[$g(...)$] {\tt conductance}
\end{itemize}

\paragraph{User Layer Description}
\begin{verbatim}
<component  name="h current">
  <definition language="NineML">
    http://www.NineML.org/1.0/currents/generic_current.9ml
  </definition>
  <properties>
    <quantity name="reversePotential">
      <value>
        <scalar>-30.0</scalar>
        <unit>mV</unit>
      </value>
    </quantity>
    <quantity name="peakConductance">
      <value>
        <scalar>1.0</scalar>
        <unit>mS</unit>
      </value>
    </quantity>
    <quantity name="conductance">
      <reference>h conductance</reference>
    </quantity>
    <quantity name="voltage">
       <link>
         <from>value</from>
         <to>properties.quantity[@name=conductance].reference.properties.quantity[@name="voltage"].value</to>
       </link>
       <value>external</value>
    </quantity>
  </properties>
</component>
\end{verbatim}
Here we link the membrane potential within the current description to the membrane
potential in the conductance description. This is an example of hierarchical linking,
since the membrane potential is not defined in the current but rather will be given
by the neuron in the later description.

\issue{Would it be better to refer to local membrane potential through
{\tt properties.quantity[@name="voltage"].value}? Usually we provide the path from
the component level...}

\subsubsection{More subtypes here}

\subsection{Spike-Producing Components}

Spike-producing components are basically neuronal models. They consist of
a definition of internal dynamics, set of properties plus a block of arbitrary
number of currents described in section~\ref{currents}.

\subsubsection{Extra Currents In Spike-Producing Nodes}

Here is an example of a spike-producing component from \cite{hill-tononi-05}.
\begin{verbatim}
<component name="Example cell">
  <definition language="NineML">
    http://www.NineML.org/neurons/hill-tononi.xml
  </definition>
  <properties>
    <quantity name="membraneCapacitance">
      <value>
        <scalar>1</scalar>
        <unit>uF per cm sq</unit>
      </value>
    </quantity>
    <quantity name="voltage">
      <value>
        <scalar>-60</scalar>
        <unit>mV</unit>
      </value>
    </quantity>
  <properties>
  <currents>
    <current>
      <reference>h current</reference>
      <link>
        <from>properties.quantity[@name="voltage"].value</from>
        <to>reference.properties.quantity[@name="voltage"].value</to>
      </link>
    </current>
    <current>
      <reference>Injection current</reference>
    </current>
  </currents>
</component>
\end{verbatim}
This example refers to both injected current and the h-current defined in
section~\ref{currents}. Note that it links the membrane potential of a neuron
(property {\tt voltage}) to a membrane potential of an h-current thus
completing the definition.

\issue{It is probably better to describe injected currents in a fashion similar
to other inputs, so that the neuronal description only includes internal currents
specific for the cell type.}

\subsubsection{Simple Poisson Spike-Producing Node}

An approximation of the neuron used as a source of input in many models
(e.g. \cite{brunel-00},\\ \cite{morrison-etal-07}).

\paragraph{General Definition}
Poisson generator is described as homogeneous Poisson process where the
waiting time till each next spike is distributed exponentially
\begin{equation}
P(T_{next})=-\nu t
\label{poisson}
\end{equation}
and has a single parameter $\nu$ in the equation (\ref{poisson}) that
corresponds to user layer name {\tt rate}.

\paragraph{User Layer Description}
\begin{verbatim}
<component name="E_ext">
  <definition language="NineML">
    http://www.NineML.org/neurons/Poisson.9ml
  </definition>
  <properties>
    <quantity name="rate">
      <value>
        <scalar>unclear</scalar>
        <unit>unit per s</unit>
      </value>
    </quantity>
  </properties>
</component>
\end{verbatim}
This code refers to the abstraction layer definition of the Poisson spiker and
sets the parameter {\tt rate}.

\subsubsection{Leaky Integrate-and-Fire Defined Through Membrane
Time Constant}
\label{IaF_tau_Desc}

\paragraph{General Definition}
Proposed type name {\begin{verbatim}IaF_tau\end{verbatim}} represents IaF
component with default dynamics described as
\begin{equation}
\tau\frac{dV}{dt}=-V
\label{IaF_tau}
\end{equation}
If voltage $V>\theta$ the neuron emits a spike, resets the voltage to
$V_r$, and enters a refractory state for $t_r$. Correspondence between
properties and names:
\begin{itemize}
\item[$V$] {\tt voltage}
\item[$t_r$] {\tt refractoryTime}
\item[$V_r$] {\tt resetPotential}
\item[$\theta$] {\tt threshold}
\item[$\tau$] {\tt membraneConstant}
\end{itemize}
Equation (\ref{IaF_tau}) has to be extended to allow addition of all
currents the neuron might have. Thus the place holders for these currents
are provided:
\begin{equation}
\tau\frac{dV}{dt}=-V+\frac{\tau}{C}\sum_i{I_i}
\label{IaF_tau_full}
\end{equation}
which adds a property
\begin{itemize}
\item[$C$] {\tt membraneCapacitance}
\end{itemize}
and an arbitrary number of currents.

\issue{Something has to be done with this property to distinguish it
from properties above that have to be always present. This one is
only necessary if currents are present, maybe bind it to currents
block?}

Delta synapses that operate in terms of voltage steps do not produce
currents, so they have to be added separately. Again, place holders
are provided in the equation
\begin{equation}
V=V+\sum_j{V_j}
\label{IaF_tau_delta}
\end{equation}

\paragraph{User Layer Description}
\begin{verbatim}
<component name="E">
  <definition language="NineML">
    http://www.NineML.org/neurons/IaF_tau.9ml
  </definition>
  <properties>
    <quantity name="membraneConstant">
      <value>
        <scalar>20</scalar>
        <unit>ms</unit>
      </value>
    </quantity>
    <quantity name="threshold">
      <value>
        <scalar>20</scalar>
        <unit>mV</unit>
      </value>
    </quantity>
    <quantity name="refractoryTime">
      <value>
        <scalar>2</scalar>
        <unit>ms</unit>
      </value>
    </quantity>
    <quantity name="resetPotential">
      <value>
        <scalar>10</scalar>
        <unit>mV</unit>
      </value>
    </quantity>
    <quantity name="voltage">
      <value>
        <scalar>-60</scalar>
        <unit>mV</unit>
      </value>
    </quantity>
  </properties>
</component>
\end{verbatim}

Please note that description of the neuronal component above defines a prototypical
neuron which can be reused multiple times within the network. As a consequence,
multiple connectivity patterns can be applied to this neuron, which results in
a different set of synaptic inputs. Therefore, no description of these
inputs shall be provided in the spike-producing component, they shall be
described at the connectivity component. Simulation software shall take care of
complete construction of each neuron by analyzing both levels of description.

\subsubsection{Leaky Integrate-and-Fire Defined Through Leakage
Conductance}

\paragraph{General Definition}
Proposed type name {\begin{verbatim}IaF_gL\end{verbatim}} represents IaF
component with default dynamics described as
\begin{equation}
C_m\frac{dV}{dt}=-g_LV
\label{IaF_gL}
\end{equation}
If voltage $V>\theta$ the neuron emits a spike, resets the voltage to
$V_r$, and enters a refractory state for $t_r$. Correspondence between
properties and tag names:
\begin{itemize}
\item[$V$] {\tt voltage}
\item[$C_m$] {\tt membraneCapacitance}
\item[$t_r$] {\tt refractoryTime}
\item[$V_r$] {\tt resetPotential}
\item[$\theta$] {\tt threshold}
\item[$g_L$] {\tt leakageConductance}
\end{itemize}
Equation (\ref{IaF_gL}) has to be extended to allow addition of all
currents the neuron might have. Thus the place holders for these currents
are provided:
\begin{equation}
C_m\frac{dV}{dt}=g_L(E_L-V)+\sum_i{I_i}
\label{IaF_gL_full}
\end{equation}

\issue{I added $E_L$ in this equation but no tag so far is provided for this
property.}

Delta synapses that operate in terms of voltage steps do not produce
currents, so they have to be added separately. Again, place holders
are provided, but this time they have to operate not on the derivative
of voltage but on voltage itself as in the equation
\begin{equation}
V=V+\sum_j{V_j}
\label{IaF_gL_delta}
\end{equation}

\paragraph{User Layer Description}
\begin{verbatim}
<component name="E">
  <definition language="NINeML">http://www.NINeML.org/neurons/IaF_gL.xml</definition>
  <properties>
  ...
  </properties>
  <currents>
    ... (if needed)
  </currents>
</component>
\end{verbatim}

\subsubsection{Izhikevich Neuron}

\paragraph{General Definition}
Fill in.

\paragraph{User Layer Description}
\begin{verbatim}
<component name="Izhikevich">
  <definition language="NineML">
    http://www.NineML.org/1.0/neurons/izhikevich.9ml
  </definition>
  <properties>
    <quantity name="a">
    <value>
      <scalar>10</scalar>
      <unit>none</unit>
    </value>
    </quantity>
    <quantity name="b">
    <value>
      <scalar>10</scalar>
      <unit>none</unit>
    </value>
    </quantity>
    <quantity name="c">
    <value>
      <scalar>10</scalar>
      <unit>none</unit>
    </value>
    </quantity>
    <quantity name="d">
    <value>
      <scalar>10</scalar>
      <unit>none</unit>
    </value>
    </quantity>
    <quantity name="V">
    <value>
      <scalar>0</scalar>
      <unit>mV</unit>
    </value>
    </quantity>
    <quantity name="u">
    <value>
      <scalar>10</scalar>
      <unit>none</unit>
    </value>
    </quantity>
  </properties>
</component>
\end{verbatim}

\subsubsection{More subtypes here}

\subsection{Plasticity Components}
\label{plasticity}

Plasticity components handle the synaptic weight and its possible modification.
Note that synaptic weight is a dimensionless quantity and as such does not
include units.

\subsubsection{Static Synapse}

Often the synapses in various models are held at fixed weights. This
is achieved with static synapse descriptions.

\paragraph{General Definition}

\paragraph{User Layer Description}
\begin{verbatim}
<component name="Excitatory">
  <definition language="NINeML">
    http://www.NINeML.org/weights/Static.xml
  </definition>
  <properties>
    <quantity name="weight">
      <value>
        <scalar>45.61</scalar>
        <unit>none</unit>
      </value>
    </quantity>
  </properties>
</component>
\end{verbatim}

\subsubsection{Power Law STDP Synapse}

\issue {Add discussion of spike sampling regimes (all, nearest N, etc).}

\paragraph{General Definition}

The definition follows \cite{morrison-etal-07}:
\begin{equation}
\begin{array}{rclcl}
\Delta w_+&=&\lambda w_0^{1-\mu}w^{\mu}e^{-\frac{|\Delta t|}{\tau_+}} &if& \Delta t>0\\
\Delta w_-&=&-\lambda \alpha we^{-\frac{|\Delta t|}{\tau_-}} &if& \Delta t<0
\end{array}
\label{powerSTDP}
\end{equation}

\issue {Add a bit on how it works, what makes it different from
additive and multiplicative rules.}

Correspondence between properties and tag names:
\begin{itemize}
\item[$w$] {\tt weight}
\item[$w_0$] {\tt referenceWeight}
\item[$\tau_+$] {\tt potTimeConst}
\item[$\tau_-$] {\tt depTimeConst}
\item[$\mu$] {\tt power}
\item[$\lambda$] {\tt learningRate}
\item[$\alpha$] {\tt asymmetry}
\end{itemize}

\paragraph{User Layer Description}
\begin{verbatim}
<component name="Plastic">
  <definition language="NineML">
    http://www.NineML.org/synapses/PowerAA_STDP.9ml</definition>
  <properties>
    <quantity name="potTimeConst">
      <value>
        <scalar>20</scalar>
        <unit>ms</unit>
      </value>
    </quantity>
    <quantity name="depTimeConst">
      <value>
        <scalar>20</scalar>
        <unit>ms</unit>
      </value>
    </quantity>
    <quantity name="power">
      <value>
        <scalar>0.4</scalar>
        <unit>none</unit>
      </value>
    </quantity>
    <quantity name="learningRate">
      <value>
        <scalar>0.1</scalar>
        <unit>none</unit>
      </value>
    </quantity>
    <quantity name="asymmetry">
      <value>
        <scalar>0.11</scalar>
        <unit>none</unit>
      </value>
    </quantity>
  </properties>
</component>
\end{verbatim}

\subsubsection{More subtypes here}

\subsection{Post-synaptic Response (PSR) Components}
\label{secSynapse}

Post-synaptic response components define the effect imposed on the post-synaptic
cell dynamics by triggering the synaptic input. This definition includes only
the shape of this effect and does not include the exact magnitude of the effect.
The magnitude is described separately through plasticity rules and synaptic
weights in section \ref{plasticity}. Note though that the units of the
effect are defined here and synaptic weights are left dimensionless. This
is done to allow same plasticity rules to operate on different types
of postsynaptic responses.

\issue{Shall the post-synaptic delay be a property of PSR components or of
connectivity components?}

NINeML defines the following post-synaptic response components:

\subsubsection{Delta PSR}

This is an approximation of synaptic function by a voltage step.

\paragraph{General Definition}

\paragraph{User Layer Description}
\begin{verbatim}
<component name="Delta PSR">
  <definition language="NINeML">
    http://www.NINeML.org/synapses/Delta.xml
  </definition>
  <properties>
    <quantity name="step">
      <value>
        <scalar>1</scalar>
        <unit>mV</unit>
      </value>
    </quantity>
  </properties>
</component>
\end{verbatim}

\issue{The actual magnitude of the step here is given by a synaptic weight,
but the units are set inside delta synapse, since weight is dimensionless.
As such the {\tt value} tag above does not have any meaning, or does it?
To take it further, a lot of PSRs describe variable that changes between
0 and 1...}

\subsubsection{Alpha Current PSR}

This is an approximation of synaptic current by an alpha function.

\paragraph{General Definition}
This current described as
\begin{equation}
I=\frac{e}{\tau_{\alpha}}te^{-\frac{t}{\tau_{\alpha}}}
\label{alphaC}
\end{equation}
where $t$ is the time since presynaptic spike and $\tau_{\alpha}$ is
a time constant of the synaptic response.
Correspondence between parameter and tag name is
\begin{itemize}
\item[$\tau_{\alpha}$] {\tt timeConstant}
\item[$t$] {\tt time}
\end{itemize}

\paragraph{User Layer Description}
\begin{verbatim}
<component name="PSC">
  <definition language="NINeML">
    http://www.NINeML.org/synapses/AlphaI.xml
  </definition>
  <properties>
    <quantity name="time">
      <value>
        <scalar>0</scalar>
        <unit>ms</unit>
      </value>
    </quantity>
    <quantity name="timeConstant">
      <value>
        <scalar>0.33</scalar>
        <unit>ms</unit>
      </value>
    </quantity>
  </properties>
</component>
\end{verbatim}

\subsubsection{Exponential Conductance PSR}

This is an approximation of synaptic conductance by an exponential function.

\paragraph{General Definition}
This conductance described as instantaneously increasing to 1.0 when the spike
arrives and decaying according to
\begin{equation}
g=e^{-\frac{t}{\tau_{\alpha}}}
\label{expC}
\end{equation}
where $t$ is the time since presynaptic spike and $\tau_{\alpha}$ is
a time constant of the synaptic response.
Correspondence between parameter and tag name is
\begin{itemize}
\item[$\tau_{\alpha}$] {\tt timeConstant}
\item[$t$] {\tt time}
\item[$g$] {\tt conductance}
\end{itemize}
An additional name {\tt reversalPotential} represents reversal potential $E_R$ of the
synapse that affects the response according to
\begin{equation}
I=g(E_R-V_m)
\label{expCurRev}
\end{equation}

\paragraph{User Layer Description}
\begin{verbatim}
<component name="ExcPSR">
  <definition language="NINeML">
    http://www.NINeML.org/synapses/ExponentialG.xml
  </definition>
  <properties>
    <quantity name="time">
      <value>
        <scalar>0</scalar>
        <unit>ms</unit>
      </value>
    </quantity>
    <quantity name="conductance">
      <value>
        <scalar>0</scalar>
        <unit>nS</unit>
      </value>
    </quantity>
    <quantity name="timeConstant">
      <value>
        <scalar>5.0</scalar>
        <unit>ms</unit>
      </value>
    </quantity>
    <quantity name="reversalPotential">
      <value>
        <scalar>0.0</scalar>
        <unit>mV</unit>
      </value>
    </quantity>
  </properties>
</component>
\end{verbatim}

\subsubsection{More subtypes here}

\subsection{Connectivity Components}

Connectivity components based on the geometry of source and/or target groups shall
include descriptions of the edge handling. All connectivity components below have
a property {\tt delay} that sets the axonal delay for the corresponding set
of projections. Connectivity component definitions in the abstraction layer are
likely to be based on Connection Set Algebra, but this does not affect
User Layer instantiations.

\subsubsection{One-to-One Projection}

A simplest way to organize the connectivity. When connecting two
populations of equal sizes one projection per neuron in both populations
is created. In the simplest case of indexed populations the projection
goes from neuron $i$ in source population to neuron $i$ in target
population.

\issue{Add here descriptions of addressing with shifts, between
populations of different sizes, maybe geometry based.}

\paragraph{General Definition}

\paragraph{User Layer Description}
\begin{verbatim}
<component name="input">
  <definition language="NINeML">
    http://www.NINeML.org/connections/One-to-one.xml
  </definition>
  <properties>
    <quantity name="delay">
      <value>
        <scalar>2</scalar>
        <unit>ms</unit>
      </value>
    </quantity>
  </properties>
</component>
\end{verbatim}

\subsubsection{Pair-based Random Projection}

\paragraph{General Definition}
Random projection defined based on the fixed probability of projection for
each possible pair of cells (one from source set, one from target set).


\paragraph{User Layer Description}
\begin{verbatim}
<component name="random">
  <definition language="NINeML">
    http://www.NINeML.org/connections/RandomPair.xml
  </definition>
  <properties>
    <quantity name="probability">
      <value>
        <scalar>0.1</scalar>
        <unit>none</unit>
      </value>
    </quantity>
    <quantity name="delay">
      <value>
        <scalar>2</scalar>
        <unit>ms</unit>
      </value>
    </quantity>
  </properties>
</component>
\end{verbatim}

\subsubsection{Number-based Random Projection}

\paragraph{General Definition}
Random projection defined based on the fixed number of sources/destinations. If
the {\tt direction} is convergent the parameter {\tt number} defines the fixed
number of sources that project to each destination cell. If the {\tt direction}
is divergent the parameter {\tt number} defines the fixed number of destinations
each source cell is projecting to.

\issue{The discussion was on the mailing list about the exact terminology for
convergent/divergent, but I am not sure we concluded the issue.}

\paragraph{User Layer Description}
\begin{verbatim}
<component name="random fan-in">
  <definition language="NINeML">
    http://www.NINeML.org/connections/NBRP.xml
  </definition>
  <properties>
    <quantity name="number">
      <value>
        <scalar>1000</scalar>
        <unit>none</unit>
      </value>
    </quantity>
    <quantity name="delay">
      <value>
        <scalar>2</scalar>
        <unit>ms</unit>
      </value>
    </quantity>
    <selection name="direction">convergent</selection>
  </properties>
</component>
\end{verbatim}
Here one of the properties is of enumerated type (direction can only be convergent or
divergent) so the tag {\tt selection} is used instead of {\tt quantity}.


\subsubsection{Fraction-based Random Projection}

\paragraph{General Definition}
Random projection defined based on the fraction of sources/destinations. If
the {\tt direction} is convergent the parameter {\tt fraction} defines the fraction
of all source cells (e.g 10\%) that project to each destination cell. If the
{\tt direction} is divergent the parameter {\tt fraction} defines the fraction of
destination cells each source cell is projecting to.

\paragraph{User Layer Description}
\begin{verbatim}
<component name="random fan-in">
  <definition language="NINeML">
    http://www.NINeML.org/connections/FBRP.xml
  </definition>
  <properties>
    <quantity name="fraction">
      <value>
        <scalar>0.1</scalar>
        <unit>none</unit>
      </value>
    </quantity>
    <quantity name="delay">
      <value>
        <scalar>2</scalar>
        <unit>ms</unit>
      </value>
    </quantity>
    <selection name="direction">convergent</selection>
  </properties>
</component>
\end{verbatim}

\subsubsection{More subtypes here}

\subsection{Space Components}

Space components define the coordinate systems used to set up the geometry of the
network. These components can be Cartesian, polar, and maybe some other coordinate
systems. More than one space components can be defined in the same network, for
example Cartesian 3D for cortical cells and polar 2D for retinal space.

\issue{In case we need rotated, translated, or scaled versions of the same
coordinate system, shall we define a new space component?}

\issue{In case user combines two models how shall the space be handled?}

\subsubsection{1D Cartesian Space}

Sets a coordinate line in 1D space.

\paragraph{General Definition}

\paragraph{User Layer Description}
\begin{verbatim}
<component name="my 1D space">
  <definition language="NINeML">http://www.NINeML.org/spaces/Cartesian1D.xml</definition>
  <properties>
    ...
  </properties>
</component>
\end{verbatim}

\issue{Could not think of any properties yet. Later we might add
rotation/translation/scale here...}

\subsubsection{2D Cartesian Space}

Sets a coordinate plane in 2D space.

\paragraph{General Definition}

\paragraph{User Layer Description}
\begin{verbatim}
<component name="my 2D space">
  <definition language="NINeML">http://www.NINeML.org/spaces/Cartesian2D.xml</definition>
  <properties>
    ...
  </properties>
</component>
\end{verbatim}

\subsubsection{3D Cartesian Space}

Sets a coordinate axes in 3D space.

\paragraph{General Definition}

\paragraph{User Layer Description}
\begin{verbatim}
<component name="my 3D space">
  <definition language="NINeML">http://www.NINeML.org/spaces/Cartesian3D.xml</definition>
  <properties>
    ...
  </properties>
</component>
\end{verbatim}

\subsubsection{2D Polar Space}

Sets a polar coordinate plane in 2D space.

\paragraph{General Definition}

\paragraph{User Layer Description}
\begin{verbatim}
<component name="my 2D polar space">
  <definition language="NINeML">http://www.NINeML.org/spaces/Polar2D.xml</definition>
  <properties>
    ...
  </properties>
</component>
\end{verbatim}

\subsubsection{3D Polar Space}

Sets a polar coordinate system in 3D space.

\paragraph{General Definition}

\paragraph{User Layer Description}
\begin{verbatim}
<component name="my 3D polar space">
  <definition language="NINeML">http://www.NINeML.org/spaces/Polar3D.xml</definition>
  <properties>
    ...
  </properties>
</component>
\end{verbatim}

\subsubsection{more subtypes here}

\subsection{Region Components}

Region component defines a finite region given a coordinate system (space). A reference to a
corresponding coordinate system as well as boundaries of a region constitute
the properties of a region component.

\subsubsection{Line Segment}

Sets a line segment in 1D space.

\paragraph{General Definition}
Properties include reference to coordinate system {\tt space}, and boundaries
{\tt lowerBound} and {\tt upperBound}

\paragraph{User Layer Description}
\begin{verbatim}
<component name="my line">
  <definition language="NINeML">
    http://www.NINeML.org/regions/Line.xml
  </definition>
  <properties>
    <space><reference>my 1D space</reference></space>
    <quantity name="lowerBound">
      <value>
        <scalar>100</scalar>
        <unit>um</unit>
      </value>
    </quantity>
    <quantity name="upperBound">
      <value>
        <scalar>500</scalar>
        <unit>um</unit>
      </value>
    </quantity>
  </properties>
</component>
\end{verbatim}
Note that the reference to coordinate system still uses a specific rather than a
generic tag.

\subsubsection{Axes Aligned Rectangle}

Sets a rectangle parallel to axes in 2D space.

\paragraph{General Definition}
Properties include reference to coordinate system {\tt space}, and boundaries
{\tt lowerBoundX}, {\tt lowerBoundY}, {\tt upperBoundX} and {\tt upperBoundY}.

\paragraph{User Layer Description}
\begin{verbatim}
<component name="my square">
  <definition language="NINeML">
    http://www.NINeML.org/regions/AARectangle.xml
  </definition>
  <properties>
    <space><reference>my 2D space</reference></space>
    <quantity name="lowerBoundX">
      <value>
        <scalar>100</scalar>
        <unit>um</unit>
      </value>
    </quantity>
    <quantity name="upperBoundX">
      <value>
        <scalar>500</scalar>
        <unit>um</unit>
      </value>
    </quantity>
    <quantity name="lowerBoundY">
      <value>
        <scalar>100</scalar>
        <unit>um</unit>
      </value>
    </quantity>
    <quantity name="upperBoundY">
      <value>
        <scalar>500</scalar>
        <unit>um</unit>
      </value>
    </quantity>
  </properties>
</component>
\end{verbatim}

\subsubsection{Axes Aligned Parallelepiped}

Sets a parallelepiped aligned with the axes in 3D space.

\paragraph{General Definition}
Properties include reference to coordinate system {\tt space}, and boundaries
{\tt lowerBoundX}, {\tt lowerBoundY}, {\tt lowerBoundZ}, {\tt upperBoundX},
{\tt upperBoundY}  and {\tt upperBoundZ}.

\paragraph{User Layer Description}
\begin{verbatim}
<component name="my block">
  <definition language="NINeML">http://www.NINeML.org/regions/AAPPD.xml</definition>
  <properties>
    <space><reference>my 3D space</reference></space>
    <quantity name="lowerBoundX">
      <value>
        <scalar>100</scalar>
        <unit>um</unit>
      </value>
    </quantity>
    ...
  </properties>
</component>
\end{verbatim}

\issue{The last two only work if space is Cartesian, they will fail for polar
spaces, but it is not shown anywhere...}

\subsubsection{Circle}

Sets a circle with center and radius in 2D space.

\paragraph{General Definition}
Properties include reference to coordinate system {\tt space}, coordinates of
the center {\tt centerX} {\tt centerY}, and {\tt radius}.

\paragraph{User Layer Description}
\begin{verbatim}
<component name="fovea">
  <definition language="NINeML">http://www.NINeML.org/regions/Circle.xml</definition>
  <properties>
    <space><reference>my 2D polar space</reference></space>
    <quantity name="centerX">
      <value>
        <scalar>100</scalar>
        <unit>um</unit>
      </value>
    </quantity>
    <quantity name="centerY">
      <value>
        <scalar>100</scalar>
        <unit>um</unit>
      </value>
    </quantity>
    <quantity name="radius">
      <value>
        <scalar>500</scalar>
        <unit>um</unit>
      </value>
    </quantity>
  </properties>
</component>
\end{verbatim}

\subsubsection{Sphere}

Sets a sphere with center and radius in 3D space.

\paragraph{General Definition}
Properties include reference to coordinate system {\tt space}, coordinates of
the center {\tt centerX}, {\tt centerY}, {\tt centerZ}, and {\tt radius}.

\paragraph{User Layer Description}
\begin{verbatim}
<component name="nucleus">
  <definition language="NINeML">http://www.NINeML.org/regions/Sphere.xml</definition>
  <properties>
    <space><reference>my 3D space</reference></space>
    <quantity name="centerX">
      <value>
        <scalar>100</scalar>
        <unit>um</unit>
      </value>
    </quantity>
    <quantity name="centerY">
      <value>
        <scalar>100</scalar>
        <unit>um</unit>
      </value>
    </quantity>
    <quantity name="centerZ">
      <value>
        <scalar>100</scalar>
        <unit>um</unit>
      </value>
    </quantity>
    <quantity name="radius">
      <value>
        <scalar>500</scalar>
        <unit>um</unit>
      </value>
    </quantity>
  </properties>
</component>
\end{verbatim}

\subsubsection{more subtypes here}

\subsection{Layout Components}
\label{layout}

Layout components define how neurons or groups of neurons are positioned within
a given region of space. Note that both discrete grids and continuous
mappings can operate on the same region, so it is possible to create a 3D
grid of neurons for one population as well randomly distributed set of
neurons from other population within the same region.

Declaring the existence of a population implicitly introduces a notion of
indexing for its elements. In general, a layout is simply a mapping from these
indices to generalized positions. For spatial layouts it should reference an
existing {\tt region} which in turn will reference some coordinate system.

\subsubsection{Unstructured Spaceless Layout}

A pseudo-layout for populations where space and ordering of cells does not matter
at all (like in \cite{brunel-00}).

\paragraph{General Definition}
This layout has no properties.

\paragraph{User Layer Description}
\begin{verbatim}
<component name="unstructured">
  <definition language="NINeML">
    http://www.NINeML.org/layouts/Unstructured.xml
  </definition>
  <properties/>
</component>
\end{verbatim}

\subsubsection{1D Spaceless Grid Layout}

Sets a uniformly spaced layout. Here the space still does not matter, but
the ordering of the cells become important.

\paragraph{General Definition}
Since only the order of the cells matters here, and the number of the cells
is defined in the population that uses this layout, there are no properties.

\paragraph{User Layer Description}
\begin{verbatim}
<component name="my uniform1D">
  <definition language="NINeML">
    http://www.NINeML.org/layouts/Grid1D.xml
  </definition>
  <properties/>
</component>
\end{verbatim}

\subsubsection{1D Grid Layout}

Sets a uniformly spaced layout in 1D region of space.

\paragraph{General Definition}
Since there is only one dimension, the cell number and the region fully define
the spacing between cells, thus {\tt region} is the only necessary property.
We might consider the case when region and spacing are given and the cell count
is calculated in the population based on this information.

\paragraph{User Layer Description}
\begin{verbatim}
<component name="my uniform1D">
  <definition language="NINeML">
    http://www.NINeML.org/layouts/Grid1D.xml
  </definition>
  <properties>
    <region>my 1D region</region>
  </properties>
</component>
\end{verbatim}

\issue:{At this point I have nothing against lumping the previous two into a single !D
grid layout, and simply providing a ``fake'' region similar to how we provide ``none''
as unit for dimensionless quantities. I think it might also allow merging multiple
populations in a single ``pseudo-space where multiple grids overlap, whether it
is useful is unclear.}

\subsubsection{2D Grid Layout}

Sets a uniformly spaced layout in 2D region of space.

\paragraph{General Definition}
Here the cell count given as the property of the population that refers to this layout
has to be a product of {\tt xCount} and {\tt yCount}.

\issue{If it is not we need to clearly define what to do with the extra neurons or spaces
in the grid.}

\paragraph{User Layer Description}
\begin{verbatim}
<component name="my uniform2D">
  <definition language="NINeML">http://www.NINeML.org/layouts/Grid2D.xml</definition>
  <properties>
    <region><reference>my 2D region</reference></region>
    <quantity name="xCount">
      <value>
        <scalar>50</scalar>
        <unit>none</unit>
      </value>
    </quantity>
    <quantity name="yCount">
      <value>
        <scalar>25</scalar>
        <unit>none</unit>
      </value>
    </quantity>
  </properties>
</component>
\end{verbatim}

\subsubsection{3D Grid Layout}

Sets a uniformly spaced layout in 3D region of space.

\paragraph{General Definition}
Here the cell count count given as the property of the population that refers to this layout
has to be a product of {\tt xCount}, {\tt yCount}, and {\tt zCount}.

\issue{If it is not we need to clearly define what to do with the extra neurons or spaces
in the grid.}

\paragraph{User Layer Description}
\begin{verbatim}
<component name="my uniform3D">
  <definition language="NINeML">
    http://www.NINeML.org/layouts/Grid3D.xml
  </definition>
  <properties>
    <region><reference>my 3D region</reference></region>
    <quantity name="xCount">
      <value>
        <scalar>50</scalar>
        <unit>none</unit>
      </value>
    </quantity>
    ...
</properties>
</component>
\end{verbatim}

\subsubsection{1D Random Uniformly Distributed Layout}

Sets a uniformly distributed random layout in 1D region of space.

\paragraph{General Definition}

\paragraph{User Layer Description}
\begin{verbatim}
<component name="my uniform1D">
  <definition language="NINeML">
    http://www.NINeML.org/layouts/Uniform1D.xml
  </definition>
  <properties>
    <region><reference>my 1D region</reference></region>
  </properties>
</component>
\end{verbatim}

\subsubsection{2D Random Uniformly Distributed Layout}

Sets a uniformly distributed random layout in 2D region of space.

\paragraph{General Definition}

\paragraph{User Layer Description}
\begin{verbatim}
<component name="my uniform2D">
  <definition language="NINeML">
    http://www.NINeML.org/layouts/Uniform2D.xml
  </definition>
  <properties>
    <region><reference>my 2D region</reference></region>
  </properties>
</component>
\end{verbatim}

\subsubsection{3D Random Uniformly Distributed Layout}

Sets a uniformly distributed random layout in 3D region of space.

\paragraph{General Definition}

\paragraph{User Layer Description}
\begin{verbatim}
<component name="my uniform3D">
  <definition language="NINeML">
    http://www.NINeML.org/layouts/Uniform3D.xml
  </definition>
  <properties>
    <region><reference>my 3D region</reference></region>
  </properties>
</component>
\end{verbatim}

\subsubsection{more subtypes here}

\section{Grouping}
\label{grouping}

One of the issues the Task Force faced was to define standardized terms
to describe network and network topology. It appeared that the use of words
such as layer, population, and such is confusing. The term `population' is
still used below to designate a collection of identical elements because
it appears the most common use of it within the Task Force.

Grouping can apply not only to cells but also to synapses, currents, and all
other components defined in the previous sections. Technically, the grouping of objects
is of two major types: \emph{\bf constructive} grouping used to {\bf create} objects
and \emph{\bf access} grouping used to define the subgroups of {\bf already created}
objects for the purposes of connectivity or monitoring.

To eliminate access groups that fully replicate corresponding constructive groups any
constructive group can be used for the access purposes. Both types of grouping allow
hierarchical build of larger groups from smaller groups. Rather than having a deep
hierarchy, however, the structure is kept flat by referencing. A group does not
contain the description of the elements that are repeated, but only refers to their
original description.

The rules of construction provided by constructive groups shall not be
executed immediately on occurrence of such a group within the description.
The process of construction shall proceed from top-most group in the hierarchy
down to component groups. This allows the flexibility of defining one constructive
group in the description and reusing it multiple times within the network.
This also allows the inclusion of developed networks in the larger models
as sub-components. The top-most group in the hierarchy is not marked anywhere
in the description, it is rather selected by the user through the software
interface. This way the user can simulate individual components of the network
for debugging and tuning purposes without modifying the rest of the model.

\subsection{Constructive Group ({\tt group})}
\label{group}

This is a two-layered structure including of a set of populations on the lower level
and connectivity between these populations together with access groups (section
\ref{accessGroup}) on the upper level. Each population describes a collection of
\emph{identical} elements. Both the number of populations and the number of elements
within population can be set to one. Processing of a constructive group by a
NINeML-compliant software shall lead to creation of instances for all elements,
define the spatial layout of the group, and provide internal connectivity if necessary.
Constructive groups use tag {\tt group} in user layer descriptions.

A group contains a user-given ID ({\tt name}). The scoping rules for these names
are described in section \ref{scope}.

An user layer description example of a three population group with 90000 cells of
the first type ({\tt E} marked by {\tt prototype} tag here), 22500 cells of the second
type ({\tt I}), and 112500 cells of the third type ({\tt E\_ext}):
\begin{verbatim}
<group name="Network">
  <population name="Exc">
    <prototype><reference>E</reference></prototype>
    <layout><reference>unstructured</reference></layout>
    <number>90000</number>
  </population>
  <population name="Inh">
    <prototype><reference>I</reference></prototype>
    <layout><reference>unstructured</reference></layout>
    <number>22500</number>
  </population>
  <population name="Ext">
    <prototype><reference>E_ext</reference></prototype>
    <layout><reference>unstructured</reference></layout>
    <number>
      <calculate>population[@name="Exc"].number+population[@name="Inh"].number</calculate>
    </number>
  </population>
  <projection>...</projection>
  ...
  <projection>...</projection>
</group>
\end{verbatim}
In all three populations {\tt prototype} tag refers to a user-given ID ({\tt name})
of the prototypical component. Note also that {\tt prototype}s in different populations
can use the same prototypical component, for example in the code above another population
consisting of a number of inhibitory cells of type {\tt I} can be added if the user
wants to define two populations of interneurons with different connectivity or
with different positional distribution.

The line
\begin{verbatim}
<number>
  <calculate>population[@name="Exc"].number+population[@name="Inh"].number</calculate>
</number>
\end{verbatim}
is an example of the syntax that allows definition of properties through
some calculation based on the values of other properties.

A collection of {\tt projection} entries describes internal connectivity within
this group according to the rules shown in section \ref{projections}.

Possibility for standard extension: technically, if the prototypes refer to
definitions of different compartments, {\tt number}s are set to one and projections
are done based on diffusion currents, this group can implement a multi-compartmental
cell or a part of such a cell.

\subsection{Hierarchy of Constructive Groups}
\label{groupHierarchy}

In the previous example group used cells as prototypes, but this is not required.
To allow the arbitrary complexity of the model, prototypes in constructive groups
can be other constructive groups. Please note that these have to be groups, not
populations within a group. An example of a constructive group with
1000 networks of the same type defined above:
\begin{verbatim}
<group name="Mega Network">
  <population name="Sub Net">
    <prototype><reference>Network</reference></reference></prototype>
    <layout><reference>unstructured</reference></layout>
    <number>1000</number>
  </population>
  <projection>...</projection>
</group>
\end{verbatim}
This example will create total of 90 millions of cells of type {\tt E}
and so forth. The set of {\tt projection} descriptions describes internal for
this group connectivity according to the rules shown in section \ref{projections}.
Note, that this connectivity is between {\tt Network}s, since within
{\tt Network} projections were defined in the description of {\tt Network}.

\subsection{Access Group ({\tt set})}
\label{accessGroup}

Access group does not define how the cells are created, but rather how they
shall be selected for the purpose of monitoring of the the activity or building
the connectivity. The tag used by NINeML for this kind of group is {\tt set}.
Any constructive group declaration also naturally defines a set of all the
cells it contains plus one set per each of the populations. Constructive group
user-given IDs as well as population user-given IDs can be used anywhere set IDs
can, without explicitly defining a set for each constructive group or its population.
Declaring a set does not add any objects or components to a model specification: it just
forms a new, possibly empty, set from those components that have already been declared.

\subsubsection{Selection by Logical Combinations of Criteria}

Within each select statement logical operations are possible. For completeness
all three (AND, OR, NOT) shall be allowed. In order to make tags more intuitive
to non-mathematical users the suggested tags are
\begin{itemize}
\item[\tt any] {for logical OR operation, where match of any of the given criteria
is sufficient to select an element;}
\item[\tt all] {for logical AND operation, where match of all given criteria is
necessary to select an element;}
\item[\tt not] {for logical NOT operation, where only elements that do not match
the criteria are selected.}
\end{itemize}

\subsubsection{Selection by a User-given ID}

Here is an example of a set where the only matching criteria are user-given IDs
({\tt name}) that can be found while descending the tree within the parent group
of the set description. It only selects internal cells and excludes external
input cells:
\begin{verbatim}
<group name="Network">
  ...
  <set name="All neurons">
    <select>
      <any>
        <equal>
          <string>population[@name]</string>
          <string>Exc</string>
        </equal>
        <equal>
          <string>population[@name]</string>
          <string>Inh</string>
        </equal>
      </any>
    </select>
</set>
  ...
</group>
\end{verbatim}
This set can be used both to record activity of integrate-and-fire cells ignoring the
Poisson spikers and select the targets of external input to this network.

\issue{Note though that in all recent examples {\tt select} is the only statement in
{\tt set} so we might just drop the {\tt select} tag:}
\begin{verbatim}
<set name="IaF neurons">
  <any>
    <equal>
      <string>population[@name]</string>
      <string>Exc</string>
    </equal>
    <equal>
      <string>population[@name]</string>
      <string>Inh</string>
    </equal>
  </any>
</set>
\end{verbatim}


Multiple syntactic constructs were discussed so far with the final choice between
them left for later discussions. When all matches are exhausted, the union of each
such binding is constructed and returned.

\subsubsection{Selection by a Cell Property}

This selection only makes sense for cells that do have certain property
defined in their parameter list. If later the model will be changed and a different
neuronal component will be used as a prototype, this select statement will become
invalid. The user shall be notified of this inconsistency by a simulation software
and resolve the problem manually. Furthermore, if the property in question has
units of measurements attached to it, the selection shall also include units.

An example syntax for selecting cells in two populations based on their
membrane potential is:
\begin{verbatim}
<select>
  <less than>
    <value>population[@name="Pyramidal Neurons"].prototype.property[@name="threshold"].value</value>
    <value>
      <scalar>15</scalar>
      <unit>mV</unit>
    </value>
  </less than>
</select>
\end{verbatim}

Note that for properties that are assigned values when components are instantiated
sets only can be created after all cells in the population are instantiated.
Same is true for the selection by geometry. Therefore any set creation shall
happen after the network is instantiated. Furthermore, since we do not support
sets based on properties that change during the simulation, the proper timing
for set creation is after instantiation and before the simulation start.
This suggests the following order: instantiate all components, instantiate all
groups, create all sets, instantiate connectivity, run the simulation.

\subsubsection{Selection by Geometry}

The main issue with this type of selection is that unlike properties, positions
are never explicitly mentioned in the description. As a result we cannot access
them using the same syntax as above. Below I show an example syntax how to
extract coordinate in some space for comparison and leave empty the part where we
supposed to extract cell coordinate.

\begin{verbatim}
<select>
  <less than>
    <distance>
      <space>my 2D space</space>
      <from>
        <posX>155</posX>
        <posY>275</posY>
      </from>
      <to><position>...</position></to>
    </distance>
    <value>150</value>
    <unit>um</unit>
  </less than>
</select>
\end{verbatim}
There should also be some mechanism that allows to verify that indeed the cells
that we are looking at are laid out on the given coordinate system.

\issue{For topographic projections the position of the source within its group
(population) partially determines the position of the target within a different
group (population). In this case we need a bit more complicated coordinate
mapping then a simple reduction to a uniform coordinate space, because we
might map between populations that are laid out in different coordinate
systems (e.g. retinal and cortical).}

\section{Projections}
\label{projections}

A {\tt projection} tag holds a description of the connectivity between two groups
of cells. Along with a user-given ID ({\tt name}) unique inside the hosting group,
each projection description contains {\tt source} and {\tt target} elements, each
of which can reference either individual cell or a set. In the latter case each
{\tt projection} tag leads to creation of a set of projections rather than a
single projection.

Projection description also includes references to a plasticity component ({\tt plasticity})
that controls the synaptic weight and post-synaptic response component ({\tt response})
that controls the influence of the input through this projection on the post-synaptic
cell dynamics. Both appear in the description of the projection rather than
neuron because the same type of neuron can use different synapses when instantiated in
different populations, similarly the same postsynaptic response can be used in multiple
projections with different plasticity rules. Instead of providing the user with all
possible pre-wired combinations, NINeML allows user to combine the
plasticity$\rightarrow$response$\rightarrow$neuron chain from a set of small
standard components.

Finally, a connectivity pattern or rule shall be set for a projection through
reference to a connectivity component. This is done with tag {\tt rule}.

Example description of the projection
\begin{verbatim}
<projection name="Inhibitory">
  <source><reference>Inhibitory cells</reference></source>
  <target><reference>Input receivers</reference></target>
  <response><reference>PSR</reference></response>
  <plasticity><reference>Plastic</reference></plasticity>
  <rule><reference>random fan-in</reference></rule>
</projection>
\end{verbatim}

\section{Structure of a User Layer Description}

In order to simplify the descriptions themselves and the mechanisms for
combining multiple components of the model the user layer description
is consisting of multiple files that define various components and
contains the syntax to import external files. The proposed syntax is
\begin{verbatim}
<import>...</import>
\end{verbatim}
The filling in the above statement can be a reference to a local file
\begin{verbatim}
<import>file://home/someuser/somedirectory/someneurons.xml</import>
\end{verbatim}
as well as a reference to some Internet database of models
\begin{verbatim}
<import>http://www.modelDB.org/somepath/someneurons.xml</import>
\end{verbatim}

\issue{Different technical approaches were proposed in Antwerp 2009, such as
using XPath or Xinclude, but we decided that these technical issues should
be investigated before the next meeting. They have not been investigated yet.}

\subsection{Scope of User Layer Descriptions}
\label{scope}

In order to avoid the collisions of user-given IDs as well as to
provide the user with the ability to redefine some of the imported
components the following set of rules is suggested:
\begin{itemize}
\item{Every identifier is considered unique only within its parent}
\item{To access the identifier a full path from the point of access
to this identifier shall be provided, e.g. accessing a threshold
parameter of a pyramidal cell in a cortical layer II from a top level
network that includes all six layers of cortex will look similar to
\begin{verbatim}
group[@name="layerII"].population[@name="pyramidal"].prototype.property[@name="threshold"].value
\end{verbatim}
Here the first level accesses the group ``layer II'', the second level
accesses the population ``pyramidal'' within the ``layer II'' group, the third
level accesses the prototypical cell description and the last level
selects threshold value from this description. The exact syntax remains to be
defined.}
\item{Any imported construct can be redefined within the importing
description by adding a new definition of the same object. For example
if the file {\tt pyramidal.xml} contains the description of a pyramidal
cell named {\tt ``Pyramidal Cell''} that the user wants to reuse with a
different parameter value the following syntax shall be used:
\begin{verbatim}
<import>
  http://www.modelDB.org/somepath/pyramidal.xml
</import>
<component name="Pyramidal Cell">
  <quantity name="threshold">
    <value>
      <scalar>-35</scalar>
      <unit>mV</unit>
    </value>
  </quantity>
</component>
\end{verbatim}
In this example all properties of the cell are imported from external
file and the value of the threshold is redefined in the local description.
Note that the above example is only valid if a component named {\tt ``Pyramidal
Cell''} was described in the imported file.}
\item{If multiple redefinitions of the same entity are found, then the most
recent one will be used. The most recent is determined by the order of
parsing of the top level model description file. Here the rule shall be
enforced that when multiple files are included all components are parsed before
all groups. This will allow a smooth transition from networks with simple
components and groups to networks where some components and some groups are redefined.
A good example is a transition from Brunel's model A to model B in the
appendix.}
\item{If several different subfields are redefined, then all of these
new definitions will be used.}
\end{itemize}

Simulation software developers can (but at this
point of time are not obliged to) optimize the NINeML code that is
output by their software by grouping all redefinitions of properties of
the same component into a single component description.

\subsection{Replacing Components of a User Layer Description}

\issue{Add here discussion with Hugo about renaming/changing model/prototype
without the necessity to modify analysis scripts.}

\section{Appendix: Formal Grammar of User Layer Descriptions}

In Antwerp 2009 we decided to add a systematic definition of user layer tags here.
To be filled.

\bibliographystyle{apalike}
\bibliography{ProposalUL}

\end{document}


Antwerp:
The composition of the populations in Marino 2005 is based on a percentage of the total
number of neurons. The language should support this representation. The connectivity
is based on a probability function dependent on the distance between neurons and includes
a mapping from cortical to visual field coordinates. We need to have a proper space
definition as well as a mechanism of conversion between space representations. The
connectivity needs to be expressed based not only on space but also on other parameters.
Boundary neurons are defined using a Poissonian mask. Parameter values need to be
specified relative to each other, like number of incoming (excitatory or inhibitory)
connections as fractions of all incoming connections. In summary, the current connectivity
concepts are not sufficient to express this model, and as a consequence the space concepts
will be re-worked, using this model as a test-case.


Padraig:
1) Specify which group/population is N and which is M (N can = M), and
the global properties of the network connection (e.g. synapse type/
global syn properties)

2) Decide which group N or M to generate from. M will ultimately be the
postsynaptic group (i.e. receive syn input), but the rest of the
connection is made from one of these, the generate start group S. The
finish group is F
Default: S = N

3) Decide on the global subset of cells in S will have some connections
of this type (I assume that N can be a compound
population/group/assembly, so this set could be "all pyramidal cells in
layer 4")
Default: all

4) Decide on the global subset of cells in F will have some connections
of this type
Default: all

5) Decide on number of connections from each cell in S to build
connections from (k in your mail below)
Default: 1

6) Decide on iteration method (make one connection on cell i in S, move
to i+1, etc., come back to i...; alternatively make all conns on cell i
then move on)
Default: first option above

For each individual connection:
7) Decide on subset of cells of F which can take this connection (can be
function of geometry of layouts of S and F & position of cell i in S)
Default: all

8) Decide which cell of this subset to pick (e.g. closest)
Default: random

9) Implement conditions on finish cell (e.g. there are a max number of
connections from any cell in S allowed, only one connection from each
cell in S is allowed, autapses aren't allowed, etc). If conditions not
satisfied, go back to 7).

10) When a connection is made calculate the connection specific
properties (e.g. weights), possibly in terms of geometry of layouts of S
and F

Robert:
However, in the latter case, I'd still vote for not
using the word "convergent" in connection with the first case above,
but rather giving it some handy name (or ontology entry) and storing
the information somewhere else that this constraint can be met with
a convergent allocation method

Hans:
In your
description, two aspects are conflated: the distribution of the **number** n of connections and
the distribtion of the **location** of sources within A. For the "common" convergent connection
as in the Brunel paper, n is fixed and sources are chosen with uniform probability.

In some cases, though, as for the Vogel & Abbotts paper, we need to retain procedural elements
in the description of connectivity, since they explicitly depend on first connecting one group,
then connecting the next based on the outcome of the first, etc.

I like the idea of a "cheat sheet", i.e., a repository of advice on how to implement certain
modeling concepts. This has several advantages:

- Modeling concepts in the language core need not be bogged down with implementation constraints.

- Over time, people can add implementation advice without having to modify the description of
concepts.

- You could have multiple pieces of advice for a single concept, e.g., implementations that are
optimal under certain conditions only, for specific hardware, etc.

Robert:
Anyway, pressing on, if, say I have n cells in population A and n in
population B, and for each cell in A I pick m targets at random, then
each cell in B receives k connections where k has a binomial distribution
P(k=r) = Binomial(n,m,r) = n!/r!(n-r)! (m/n)^(r) (1- m/n)^(n-r)

So, for the language we could call this something like a random
"Delta(m):Binomial(n,m)" network, to
indicate that the number of outgoing connections per source cell is an
integer delta function (ie, exactly m for each cell), and the number
of incoming connections per target cell has a binomial distribution
with  parameters n and m. (ignoring for now  that the total number of
outgoing and incoming connections should be the same, so these are not
entirely independent variables).

But there is something fishy about the term "Delta(m):Binomial(n,m)"
because the size of the network, n, figures weakly in the probability
distributions for the number of connections.

In particular, for large n, and m << n, Binomial(n,m,r) ~=
Poisson(m,r) = exp(-m)m^r/r! so the random convergent network is
rather close to a more general set of networks that could be called
"Delta:Poission(m)", which has the nice property that it is
independent of n.

However, you can't generate such a network by iterating over one
population and picking targets in the other, but I'm not sure that is
much of a loss either. Generating a "Delta(m):Binomial(n,m)" network
that way is a bit like generating binomial variates by repeated
sampling, which is fine, but not necessarily the fastest method. For
the Delta:Poisson case, you could, for example, put m tokens for each
source cell in a list, shuffle it, and then, for each target cell,
generate a Poisson random variate with some standard method, and take
that many tokens off the list to use as its source cells.

Now, if we can express "Delta:Poission(m)", connectivity we can
equally well express "Poission:Poisson(m)" connectivity, and, in
general, "User_defined_pdf_1:User_defined_pdf_2(params)" connectivity.
So this could be a way to parameterize the family of random networks
which includes classic convergent and divergent ones as particular
instances (particular choices for the forms of the two functions).


This all only applies if the elements of a population are
indistinguishable (targets selected with uniform probability). I don't
know how or whether it would extend to location dependent connections...

Hill@Stockholm

The discussions were then oriented toward a better understanding of the grid representation for the
network topology and the possible generalization of this representation, as well as the probability function.
It was proposed that the grid representation could be extended to a general n-dimensional approach and
that such a representation could be used for the purpose of the abstraction layer. The hexagonal packing
was also mentioned during the discussion. It was also made clear that our representation should be able to
specify that: “the synapse placement should be on segment x”. This need will be more relevant in the
future when modeling studies will take into account the precise innervations’ pattern and the statistics of
synapse placement in their models with multicompartments.

Conn Set @ Stockholm

It was made clear that we need a set of operators general enough to describe broad range of connectivity.
We all agreed that this is a powerful general concept but it needs to be constrained in order to be used in
the abstraction layer. Mikael Djurfeldt then pointed out the need of expanding the practical examples to be
able to constrain this general set of operator. For this purpose, the connection set algebra could be bounded
with NEST to increase the number of different network topology to be represented by the connection set
algebra. It was also pointed out that the current framework does not cover the possible dynamic
representation of the connectivity. This can be added by the use of function.
