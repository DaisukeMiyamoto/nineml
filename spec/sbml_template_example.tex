\documentclass{ninemlspec}
\usepackage{microtype}

%% ============================================================================
%% Description:  Documentation for sbmlpkgspec.cls
%% First author: Michael Hucka <mhucka@caltech.edu>
%% Organization: California Institute of Technology
%% Date created: September 2011
%% https://sbml.svn.sourceforge.net/svnroot/sbml/trunk/project/tex/sbmlpkgspec
%%
%% Copyright (C) 2011-2014 California Institute of Technology, Pasadena, CA.
%%
%% SBMLPkgSpec is free software; you can redistribute it and/or modify it
%% under the terms of the GNU Lesser General Public License as published by
%% the Free Software Foundation.  A copy of the license agreement is provided
%% in the file named "LICENSE.txt" included with this software distribution.
%% ============================================================================

% Here we list common SBML entities, to get consistent font & appearance.

\newcommand{\AlgebraicRule}{\textbf{\class{AlgebraicRule}}\xspace}
\newcommand{\Annotation}{\textbf{\class{Annotation}}\xspace}
\newcommand{\AssignmentRule}{\textbf{\class{AssignmentRule}}\xspace}
\newcommand{\Compartment}{\textbf{\class{Compartment}}\xspace}
\newcommand{\Constraint}{\textbf{\class{Constraint}}\xspace}
\newcommand{\Delay}{\textbf{\class{Delay}}\xspace}
\newcommand{\EventAssignment}{\textbf{\class{EventAssignment}}\xspace}
\newcommand{\Event}{\textbf{\class{Event}}\xspace}
\newcommand{\FunctionDefinition}{\textbf{\class{FunctionDefinition}}\xspace}
\newcommand{\InitialAssignment}{\textbf{\class{InitialAssignment}}\xspace}
\newcommand{\KineticLaw}{\textbf{\class{KineticLaw}}\xspace}
\newcommand{\ListOfCompartments}{\textbf{\class{ListOfCompartments}}\xspace}
\newcommand{\ListOfConstraints}{\textbf{\class{ListOfConstraints}}\xspace}
\newcommand{\ListOfEventAssignments}{\textbf{\class{ListOfEventAssignments}}\xspace}
\newcommand{\ListOfEvents}{\textbf{\class{ListOfEvents}}\xspace}
\newcommand{\ListOfFunctionDefinitions}{\textbf{\class{ListOfFunctionDefinitions}}\xspace}
\newcommand{\ListOfInitialAssignments}{\textbf{\class{ListOfInitialAssignments}}\xspace}
\newcommand{\ListOfLocalParameters}{\textbf{\class{ListOfLocalParameters}}\xspace}
\newcommand{\ListOfModifierSpeciesReferences}{\textbf{\class{ListOfModifierSpeciesReferences}}\xspace}
\newcommand{\ListOfPackages}{\textbf{\class{ListOfPackages}}\xspace}
\newcommand{\ListOfParameters}{\textbf{\class{ListOfParameters}}\xspace}
\newcommand{\ListOfReactions}{\textbf{\class{ListOfReactions}}\xspace}
\newcommand{\ListOfRules}{\textbf{\class{ListOfRules}}\xspace}
\newcommand{\ListOfSpeciesReferences}{\textbf{\class{ListOfSpeciesReferences}}\xspace}
\newcommand{\ListOfSpecies}{\textbf{\class{ListOfSpecies}}\xspace}
\newcommand{\ListOfUnitDefinitions}{\textbf{\class{ListOfUnitDefinitions}}\xspace}
\newcommand{\ListOfUnits}{\textbf{\class{ListOfUnits}}\xspace}
\newcommand{\ListOf}{\class{ListOf\rule{0.3in}{0.5pt}}\xspace}
\newcommand{\LocalParameter}{\textbf{\class{LocalParameter}}\xspace}
\newcommand{\Message}{\textbf{\class{Message}}\xspace}
\newcommand{\Model}{\textbf{\class{Model}}\xspace}
\newcommand{\ModifierSpeciesReference}{\textbf{\class{ModifierSpeciesReference}}\xspace}
\newcommand{\Notes}{\textbf{\class{Notes}}\xspace}
\newcommand{\Package}{\textbf{\class{Package}}\xspace}
\newcommand{\Parameter}{\textbf{\class{Parameter}}\xspace}
\newcommand{\Priority}{\textbf{\class{Priority}}\xspace}
\newcommand{\RateRule}{\textbf{\class{RateRule}}\xspace}
\newcommand{\Reaction}{\textbf{\class{Reaction}}\xspace}
\newcommand{\Rule}{\textbf{\class{Rule}}\xspace}
\newcommand{\SBML}{\textbf{\class{SBML}}\xspace}
\newcommand{\SBase}{\textbf{\abstractclass{SBase}}\xspace}
\newcommand{\SimpleSpeciesReference}{\textbf{\class{SimpleSpeciesReference}}\xspace}
\newcommand{\SpeciesReference}{\textbf{\class{SpeciesReference}}\xspace}
\newcommand{\Species}{\textbf{\class{Species}}\xspace}
\newcommand{\StoichiometryMath}{\textbf{\class{StoichiometryMath}}\xspace}
\newcommand{\Trigger}{\textbf{\class{Trigger}}\xspace}
\newcommand{\UnitDefinition}{\textbf{\class{UnitDefinition}}\xspace}
\newcommand{\Unit}{\textbf{\class{Unit}}\xspace}


% Macros just for this document:

\newcommand{\ninemlpkg}{\texorpdfstring{%
    \textls[-25]{\textsc{SBMLPkgSpec}}}{%
    \textsc{SBMLPkgSpec}}\xspace}
\newcommand{\ninemlpkghead}{\texorpdfstring{%
    \textls[-50]{\textsc{SBMLPkgSpec}}}{%
    \textsc{SBMLPkgSpec}}\xspace}
\newcommand{\ninemlpkgfile}{\literalFont{sbmlpkgspec.cls}\xspace}
\newcommand{\latex}{\LaTeX{}\xspace}
\newcommand{\tex}{\TeX{}\xspace}
\newcommand{\distURL}{http://sourceforge.net/projects/sbml/files/specifications/tex}
\newcommand{\srcURL}{https://sbml.svn.sourceforge.net/svnroot/sbml/trunk/project/tex/sbmlpkgspec}
\newcommand{\webURL}{http://sbml.org/Documents/Specifications/The_SBMLPkgSpec_LaTeX_class}
\newcommand{\cmd}[1]{\literalFont{\textbackslash #1}}

% Custom latex listing style, for use with the listings package.  The default
% highlights far too many things, IMHO.  This keeps it simple and only adjusts
% the appearance of comments within listings.

\lstdefinelanguage{mylatex}{%
  morekeywords={},%
  sensitive,%
  alsoother={0123456789$_},%$
  morecomment=[l]\%%
}[keywords,tex,comments]

\lstdefinestyle{latex}{language=mylatex}

% -----------------------------------------------------------------------------
% Start of document
% -----------------------------------------------------------------------------

\begin{document}

\packageTitle{\latex Class for SBML Package Specifications}
\packageVersion{Version 1.6.0}
\packageVersionDate{10 August 2014}

\title{\ninemlpkghead: A \latex Class for SBML Level~3
  \texorpdfstring{\\[3pt]}{}\mbox{Package Specification Documents}}

\author{Michael Hucka\\[0.25em]
  \mailto{mhucka@caltech.edu}\\[0.25em]
  Computing and Mathematical Sciences\\
  California Institute of Technology\\
  Pasadena, CA, USA
}

\maketitlepage
\maketableofcontents


% -----------------------------------------------------------------------------
\section{Introduction}
% -----------------------------------------------------------------------------

The \ninemlpkg document class for \latex provides a standard format for SBML
Level~3 \emph{package specification} documents.  In this document, I explain
how to use \ninemlpkg.  I assume readers are familiar with \latex.  (There are
many tutorials and books available on \latex, and readers should have no
trouble finding resources if needed.)

The document before you is itself formatted using the \ninemlpkg class, with a
minor change to omit some SBML Level~3 package-specific text normally placed
on the front page.  (For example, the front page of this document does not
announce it is an ``SBML Level 3 Package Specification''.)  In other
respects, ``what you see is what you get''---this is the appearance of an
SBML Level~3 package specification document when it is formatted with
\ninemlpkg.


\subsection{Prerequisites and installation}

\tab{where} lists the Internet locations from where you may obtain
\ninemlpkg. The class itself consists of one file, \ninemlpkgfile, and a
subdirectory named ``\texttt{logos}''.  It also comes with some accompanying
documentation (specifically, the file you are reading, along with the source
files used to produce it).

\begin{table}[hb]
  \begin{edtable}{tabular}{ll}
    \toprule
    \textbf{Item} & \textbf{Location} \\
    \midrule
    Distribution archive & \url{\distURL}\\
    Web page		 & \url{\webURL}\\
    Source tree (SVN)    & \url{\srcURL}\\
    \bottomrule
  \end{edtable}
  \caption{Where to find \ninemlpkg on the Internet.}
  \label{where}
\end{table}

The use of \ninemlpkg should require only a recent and relatively complete
installation of \LaTeXe.  I developed and tested this document class with the
TeX Live 2011 distribution on a Mac~OS~X 10.6.x system, and it is has been
reported to work with TeX Live on Windows and Ubuntu Linux.  (For Ubuntu,
make sure to install the following packages: ``\texttt{texlive}'',
``\texttt{texlive-latex-extra}'', ``\texttt{texlive-humantities}'', and
``\texttt{texlive-fonts-extra}''.)  To use \ninemlpkg, you will need to inform
your copy of \latex where to find the file \ninemlpkgfile and its accompanying
subdirectory ``\texttt{logos''}.  This can be done in a variety of ways.
Here are two common ones:

\begin{itemize}

\item \emph{Per-document installation}.  This is arguably the simplest
  approach.  Download \ninemlpkg from a distribution site (see \tab{where}),
  copy the contents (specifically, \ninemlpkgfile and the folder named
  ``\texttt{logos}'') to the folder where you keep the other files for the
  SBML Level~3 package specification you are authoring, and you are done.
  The next time you run \latex in that folder (assuming you declare the
  document class as explained in \sec{usage}), it will find the \texttt{.cls}
  file in the current directory and be on its merry way.

\item \emph{``Central'' installation}.  In this approach, you install
  \ninemlpkgfile in a folder where you keep other \latex class files, and
  configure your copy of \latex to find things there.  Configuring a \tex
  system in this way on Unix-type systems (Linux, etc.)\ usually requires
  setting the environment variable \texttt{TEXINPUTS} and possibly others.
  Please consult the documentation for your \tex installation to determine
  how to do this.

\end{itemize}


\subsection{Special notation in this document}

Some paragraphs \notice in this document include a hand pointer in the
left margin (illustrated at the left).  These are meant to call attention
to paragraphs containing significant points that may be too easily missed
during the first reading.  Readers may wish to revisit those paragraphs
once they are actually using \ninemlpkg in practice.


\subsection{Where to send bug reports and feedback}

Please report problems you encounter with \ninemlpkg.  You can contact the
author directly, at the email address given on the cover page, or you can
file a bug report using the tracker at \url{http://sbml.org/issue-tracker}.


% -----------------------------------------------------------------------------
\section{Creating documents with \ninemlpkghead}
\label{usage}
% -----------------------------------------------------------------------------

This section provides a summary of the main features and capabilities of
\ninemlpkg, and serves as a guide to getting started with this \latex class.


\subsection{Basic document structure}
\label{basic-structure}

The following fragment illustrates the basic structure of a simple input
file.

\begin{example}[style=latex]
\documentclass{sbmlpkgspec}
\begin{document}

\packageTitle{Example}
\packageVersion{Version 1 (Draft)}
\packageVersionDate{14 August 2011}
\packageGeneralURL{http://sbml.org/Documents/Specifications/Example}
\packageThisVersionURL{http://sbml.org/Documents/Specifications/Example_14_August_2011}

\author{Michael Hucka\\[0.25em]
  \mailto{mhucka@caltech.edu}\\[0.25em]
  Computing and Mathematical Sciences\\
  California Institute of Technology\\
  Pasadena, CA, USA
}

\maketitlepage
\maketableofcontents

\section{...}
...
\end{document}
\end{example}

The fragment above illustrates the general structure expected by \ninemlpkg.
First, several commands all beginning with the characters \cmd{package} set
various internal variables that are used by \ninemlpkg to produce the final
package specification document.  For example, there is a title for the
package (\cmd{packageTitle\{\}}), a version number for the package
(\cmd{packageVersion\{\}}), and more.  Next, the author is declared.  After
that, the commands \cmd{maketitlepage} and \cmd{maketableofcontents} instruct
\latex to produce a title page and table of contents, respectively.  Then
comes the real body of the document, with section headings and so on.
Finally, the document is closed with the standard \latex command
\cmd{end\{document\}}.

If \notice your document is a draft version, make sure to add the special
argument \texttt{[draftspec]} to the \cmd{documentclass} command.  This
causes the front page of your document to have the word ``DRAFT'' printed
on it in large gray type, and for every page footer to mention ``(DRAFT)''
in it.  Here is an example of how to use this option:

\begin{example}[style=latex]
\documentclass[draftspec]{sbmlpkgspec}
\end{example}

Not shown here, but useful to know, is that \ninemlpkg also provides a command
for putting a prominent notice on the front page.  Writing
\cmd{frontNotice\{}\emph{text}\texttt{\}} will put \emph{text} in the middle
of the front page, in red.  It is useful for warning readers of your document
about known issues, for example if the document is a work in progress.

\ninemlpkg does not define special commands for formatting author information
beyond the \cmd{author} command (which is actually a standard \latex command
from the \texttt{article} document class).  It does, however, provide the
command \cmd{mailto}, which is designed to turn email addresses into
\texttt{mailto:} hyperlinks.  Any additional formatting of author and
affiliation information is up to you, using standard \latex commands.  Of
course, there are many times when multiple authors \emph{are} involved, so it
is useful to have an approach for handling that situation.  An easy approach
is to embed a \texttt{tabular} environment inside the \cmd{author} command.
The following is an example taken directly from an actual SBML Level~3
package specification document:

\begin{example}[style=latex]
\author{%
  \begin{tabular}{c>{\hspace{20pt}}c}
    Lucian P. Smith                     & Michael Hucka\\[0.25em]
    \mailto{lpsmith@u.washington.edu}   & \mailto{mhucka@caltech.edu}\\[0.25em]
    Department of Bioengineering        & Computing and Mathematical Sciences\\
    University of Washington            & California Institute of Technology\\
    Seattle, WA, US                     & Pasadena, CA, US\\
  \end{tabular}}
\end{example}

A final point about \ninemlpkg is worth mentioning right at the outset.  When
you run \latex (typically using the \texttt{pdflatex} variant) and look at
the output, you will often find that section references, page references, and
line numbers do not get refreshed correctly after one invocation of the
command.  With \ninemlpkg, you will typically have to run the command not
twice, but \notice \emph{three} times, to get the correct, final numbers,
because it uses the \texttt{varioref} package.  This is typically not a
problem during actual writing; the edit-run-preview cycle is such that you
usually only care to see the results of content changes, and for that,
running \latex only once is enough, even if it leaves reference unadjusted.
However, when you \emph{are} interested in checking (e.g.) figure and table
references, then it is important to keep in mind the fact that you need to
run \latex three times in succession to get all the reference updates to
propagate through.  The rule of thumb is: if you run \latex and the
references look wrong, run it again.


\subsection{Tables and figures}
\label{about-tables}

\ninemlpkg preloads the \latex \texttt{graphicx} and \texttt{xcolor}
packages, giving authors immediate access to the functionality of these
extensions without having to include them manually.  For example, if you
had an illustration in a file named ``\texttt{example.pdf}'', the following
fragment would generate a simple figure containing it:

\begin{example}[style=latex]
\begin{figure}
  \includegraphics{example}
  \caption{The figure caption.}
  \label{fig:example}
\end{figure}
\end{example}

The \latex \texttt{graphicx} package is extremely powerful; it offers many
options for controlling the size/scale and other characteristics of
graphics files.  Please refer to its documentation for help on how to use
it fully.

To produce pleasing-looking documents, I suggest you create your figures
using 8~point Helvetica for the text font.  \ninemlpkg redefines figures and
tables to use Helvetica as the font family and an 8~point size by default, so
creating figures with matching characteristics will lead to more
consistent-looking documents.  The stylistic choices here were made not
solely for aesthetic reasons; the tighter letter spacing of the sans serif
font, and the smaller font size, makes it easier for authors to fit material
into tables and figures.  The specific choice of Helvetica is also driven in
part by consideration of the tools available to authors.  In particular, it
is today common to find online drawing tools that offer Helvetica (or at
least the similar-looking, albeit inferior, Arial) as a font choice.

\ninemlpkg also preloads the \texttt{booktabs} package.  This provides
\cmd{toprule}, \cmd{midrule} and \cmd{bottomrule} (among others), which can
be used to produce attractive tables.  The following text is what produced
\fig{where}:

\vspace*{-0.25ex}
\begin{example}[style=latex]
\begin{table}[hb]
  \begin{edtable}{tabular}{ll}                 % From the lineno package; see text, Section 2.2
    \toprule                                   % From booktabs -- generates line at top
    \textbf{Item}        & \textbf{Location} \\
    \midrule                                   % From booktabs -- generates middle line
    Distribution archive & \url{\distURL} \\
    Web page             & \url{\webURL} \\
    Source tree (SVN)    & \url{\srcURL} \\
    \bottomrule                                % From booktabs -- generates line at bottom
  \end{edtable}
  \caption{Where to find \ninemlpkg on the Internet.}
  \label{where}
\end{table}
\end{example}

In the case of long tables, readability is often enhanced by adding a
background color to every other row.  Once again, \ninemlpkg preloads a
\latex package (in this case, \texttt{xcolor} with the \texttt{[table]}
option) that provides a convenient facility for automatically coloring
alternate rows in a table.  Although many variations are possible, for
consistency between SBML package specification documents, I recommend using
one in particular:

\begin{example}[style=latex]
\rowcolors{2}{ninemlrowgray}{}
\end{example}

Simply insert the text above after the opening \cmd{begin\{table\}} of your
table, and proceed as usual.  The result is demonstrated in \fig{sbmlcore},
which was produced using the following sequence:

\begin{example}[style=latex]
\begin{table}[hbt]
  \rowcolors{2}{ninemlrowgray}{}
  \begin{edtable}{tabular}{ll}|\vspace*{-0.4ex}|
  ...|\vspace*{-0.4ex}|
  \end{edtable}
\end{table}
\end{example}
  
Note \notice that tables are \emph{not} defined by \ninemlpkg to use
alternate-row background coloring by default, because in some situations
(such as short tables, or tables containing color), alternate row coloring is
unnecessary and distracting.  You must add the \cmd{rowcolors} command
manually, where it's appropriate.

Finally, \ninemlpkg redefines the table and figure environments to place
contents inside a \latex \texttt{centering} environment, causing the content
to be centered on the page.  You do not need to add centering commands
yourself.


\subsection{Cross-references to tables, figures and sections}
\label{references}

To refer to figures, tables, sections and other elements in your document,
please use the special commands listed in \tab{ref-commands} instead of
writing the usual idioms ``Figure\textasciitilde\textbackslash
\texttt{ref\{...\}}''.  The commands in \tab{ref-commands} will produce
\emph{both} item number and page references that are automatically
hyperlinked to the appropriate locations in the finished document; they will
also take care of adding ties in the appropriate places for you, and they use
the \texttt{vref} command from the package \texttt{varioref} (instead of the
regular \latex \texttt{ref}) to vary the text description used in page
references.

\begin{table}[h]
  \vspace*{-1ex}
  \begin{edtable}{tabular}{lll}
    \toprule
    \textbf{Command} & \textbf{Purpose} & \textbf{Example output}\\
    \midrule
    \texttt{\textbackslash fig\{}\emph{label\,}\texttt{\}}
    & Figure reference
    & Figure X on page Y\\
    \texttt{\textbackslash tab\{}\emph{label\,}\texttt{\}}
    & Table reference
    & Table X on page Y\\
    \texttt{\textbackslash sec\{}\emph{label\,}\texttt{\}}
    & Section reference
    & Section X on page Y\\
    \texttt{\textbackslash apdx\{}\emph{label\,}\texttt{\}}
    & Appendix reference
    & Appendix X on page Y\\
    \bottomrule
  \end{edtable}
  \caption{Commands for referring to figures and other entities in an
    \ninemlpkg document.  Use the commands with an argument consisting of the
    label being referenced.  For example: \cmd{fig\{myfig\}}.}
  \label{ref-commands}
\end{table}

The \ninemlpkg class also defines starred versions of the commands, that is,
\texttt{\textbackslash fig*\{}\emph{label\,}\texttt{\}},
\texttt{\textbackslash tab*\{}\emph{label\,}\texttt{\}},
\texttt{\textbackslash sec*\{}\emph{label\,}\texttt{\}}, and
\texttt{\textbackslash apdx*\{}\emph{label\,}\texttt{\}}.  These are useful
when the item being referenced is located on another page, and you want to
refer to it more than once from the text of the same paragraph. The regular
version of the commands such as ``\texttt{\textbackslash
  fig\{}\emph{label\,}\texttt{\}}'' will always produce a page reference
(e.g., ``see Figure~2.5 on the following page''), which becomes rather
tedious to read if there is more than one occurrence of it in the same
paragraph.  \notice To avoid that, use the normal version of the command the
first time you need it in a paragraph, and then use the starred version on
all subsequent occasions within the same paragraph.

It is worth noting that all the commands are clever enough to avoid writing
``... on page Y'' when the item in question is on the same page as the
reference.  The commands will write only ``Figure~X'' in that situation
automatically.

To state a range (e.g., to produce the text ``Section X to Y''), use the
command \texttt{\textbackslash
  vrefrange\{}\emph{label1\,}\texttt{\}\{}\emph{label2\,}\texttt{\}}, where
\emph{label1} and \emph{label2} are the labels of the starting and ending
items.  These can be figures, sections, etc.


\subsection{Hyperlinks}
\label{hyperlinks}

In the example of \sec{about-tables}, you may have noticed the use of the
command \cmd{url\{\}}.  That command comes as part of the \latex
\texttt{hyperref} package, and like the other packages mentioned in this
section, it is also preloaded by \ninemlpkg.  It provides a number of
facilities that are used to implement features of \ninemlpkg.  \tab{hyperref}
lists some commands you may find useful in writing SBML specification
documents.

\begin{table}[h]
  \begin{edtable}{tabular}{lll}
    \toprule
    \textbf{Command}		 & \textbf{Purpose} \\
    \midrule
    \cmd{url\{}\emph{URL\,}\texttt{\}}
    &  Produce a hyperlinked reference to \emph{URL}\\
    \cmd{nolinkurl\{}\emph{URL\,}\texttt{\}}
    & Format \emph{URL} in the same way as \texttt{\textbackslash url}, without making it a hyperlink\\
    \cmd{href\{}\emph{URL\,}\texttt{\}\{}\emph{text}\texttt{\}} 
    & Make \emph{text} a hyperlink to \emph{URL}\\
    \bottomrule
  \end{edtable}
  \caption{Commands provided by \texttt{hyperref} for creating hyperlinks.}
  \label{hyperref}
\end{table}


\subsection{Examples and literal text}
\label{example-env}

A document about file formats and programming often includes passages meant
to be literal text.  Conventionally, these are typeset in a monospaced type
face resembling the output of typewriter.  There are two ways of
accomplishing this for SBML package specifications.  The first is to use
\latex's standard \mbox{\texttt{\textbackslash
    texttt\{}\emph{text}\texttt{\}}} command.  This causes ``\emph{text}'' to
be output in a fixed-width font, like so: ``\texttt{text}''.

The second method is to use an environment defined by \ninemlpkg and
implemented using the \latex package \texttt{listings}.  This environment is
called \texttt{example}.  Wrapping any text with \cmd{begin\{example\}} and
\cmd{end\{example\}} will cause the text to be output by itself with a gray
box behind it, as in this example:

\begin{example}
This is an example of content placed within an "example" environment.
\end{example}

The \texttt{example} environment is particularly powerful.  Anything placed
inside it will be taken literally---even \latex commands will be ignored,
except for \cmd{end\{example\}}.  In fact, the entire contents of the
document fragment shown in \sec{basic-structure}, including the
\cmd{begin\{document\}} and \cmd{end\{document\}}, were all left unchanged
within the \texttt{example} environment used to produce the example.  Of
course, sometimes you \emph{do} want an embedded \latex command to be
interpreted inside the \texttt{example} environment.  For those situations,
surround the \latex sequence with vertical bar (\texttt{|}) characters.  The
vertical bar is defined as the escape character for the \texttt{example}
environment.

Finally, for those cases when it is more convenient to put the example
contents in a separate file, \ninemlpkg provides the command
\mbox{\cmd{exampleFile\{}\emph{file}\texttt{\}}}.  It will insert the
contents of \emph{file} at the point where it is invoked in the text, and
format the contents in the same style as the \texttt{example} environment.


\subsection{Color}

To increase consistency between SBML specification documents, \ninemlpkg
defines custom color names that may be used with commands such as
\texttt{color}, \texttt{colorbox}, etc.  \tab{colors} lists these color
names.  Some of the colors in \tab*{colors} are used by \ninemlpkg itself, for
design elements such as section dividers and background colors.  Others
sometimes prove useful in different contexts of a specification document, and
still others are colors that were used in past SBML documentation and are
retained in case they prove useful again in the future.

When \notice a specification document is a revision of a previous document,
it is the convention in SBML documentation to indicate text changes by
coloring them in red.  For this purpose, \ninemlpkg defines a command and an
environment.  The command \mbox{\texttt{\textbackslash
    changed\{}\emph{text}\texttt{\}}} causes \emph{text} to be written in red
(or more precisely, the color named \val{sbmlchangedcolor}, which is defined
by \ninemlpkg as a maroon red), \changed{like this}.  It can be used for long
stretches of text and include embedded spaces and formatting
commands. Alternatively, for coloring even longer stretches of text and
multiple paragraphs, you may prefer to use the environment
\texttt{blockChanged}.

\newcommand{\drawcol}[1]{\colorbox{#1}{\color{#1}\rule{10ex}{1ex}}}

\begin{table}[hbt]
  \renewcommand{\arraystretch}{1.05}
  \begin{edtable}{tabular}{lllll}
    \toprule
    \textbf{Color name}		& \textbf{Color sample} 	& \multicolumn{3}{c}{\textbf{RGB color value}} \\
    \midrule
    \texttt{ninemlblue}		& \drawcol{ninemlblue}		& red = 0.08	& blue = 0.51	& green = 0.77 \\
    \texttt{ninemlgray}		& \drawcol{ninemlgray}		& red = 0.7	& blue = 0.7	& green = 0.7 \\
    \texttt{ninemlrowgray}	& \drawcol{ninemlrowgray}		& red = 0.94	& blue = 0.94	& green = 0.94 \\
    \texttt{sbmlchangedcolor}	& \drawcol{sbmlchangedcolor}	& red = 0.69	& blue = 0.19	& green = 0.376 \\
    \texttt{sbmlnormaltextcolor}& \drawcol{sbmlnormaltextcolor}	& red = 0.27	& blue = 0.27	& green = 0.27 \\
    \texttt{extremelylightgray}	& \drawcol{extremelylightgray}	& red = 0.97	& blue = 0.97	& green = 0.97 \\
    \texttt{veryverylightgray}	& \drawcol{veryverylightgray}	& red = 0.95	& blue = 0.95	& green = 0.95 \\
    \texttt{verylightgray}	& \drawcol{verylightgray}	& red = 0.9	& blue = 0.9	& green = 0.9 \\
    \texttt{lightgray}		& \drawcol{lightgray}		& red = 0.8	& blue = 0.8	& green = 0.8 \\
    \texttt{mediumgray}		& \drawcol{mediumgray}		& red = 0.5	& blue = 0.5	& green = 0.5 \\
    \texttt{darkgray}		& \drawcol{darkgray}		& red = 0.3	& blue = 0.3	& green = 0.3 \\
    \texttt{almostblack}	& \drawcol{almostblack}		& red = 0.22	& blue = 0.22	& green = 0.22 \\
    \texttt{darkblue}		& \drawcol{darkblue}		& red = 0.1	& blue = 0.4	& green = 0.55 \\
    \texttt{mediumgreen}	& \drawcol{mediumgreen}		& red = 0.1	& blue = 0.6	& green = 0.3 \\
    \texttt{lightyellow}	& \drawcol{lightyellow}		& red = 0.98	& blue = 0.94	& green = 0.7 \\
    \texttt{verylightyellow}	& \drawcol{verylightyellow}	& red = 0.97	& blue = 0.95	& green = 0.85 \\
    \bottomrule
  \end{edtable}
  \caption{Color names defined by \ninemlpkg.  These names may be used in
    addition to the names of colors defined by the \latex package
    \texttt{xcolor}.}
  \label{colors}
\end{table}

The insertion of new sections in old documents presents another challenge:
the section \emph{numbers} of all sections after the inserted text will end
up changed even if the text of those sections is unchanged.  This produces
misleading results: section $X.Y$ may appear in both the old and new versions
of a document, but may not have the same content because the old $X.Y$ became
$X.(Y+1)$ in the new version.  For this situation, \ninemlpkg defines special
sectioning commands that format only the section numbers in red, leaving the
rest of the heading unchanged.  The commands have the suffix
\texttt{Changed}, e.g., \mbox{\texttt{\textbackslash
    sectionChanged\{}\emph{text}\texttt{\}}}, \mbox{\texttt{\textbackslash
    subsectionChanged\{}\emph{text}\texttt{\}}}, etc.  They should be used
for all (sub/subsub)section headings \emph{after} a new section.  In other
words, for every subsection after a newly-inserted section $X.Y$, use
\mbox{\texttt{\textbackslash subsectionChanged\{}\emph{text}\texttt{\}}}
until the next higher-level \mbox{\texttt{\textbackslash
    section\{}\emph{text}\texttt{\}}}.


\subsection{Commands for SBML constructs}

SBML defines a number of commands for referring to objects defined in the
main SBML specifications, such as \Species, \Reaction, etc.  Within a given
main SBML specification document---which are all in PDF format---the mention
of an object name is hyperlinked to the definition of that object elsewhere
in the document, such that clicking on the name causes the PDF reading
program to jump to the definition in the file.  Unfortunately, while it is
technically possible to hyperlink from SBML package documents to specific
parts within an external PDF file located somewhere on the Internet, the
result is more confusing and annoying than helpful.

\ninemlpkg therefore defines two sets of commands: one set predefines each of
the SBML Level~3 Core object names, but without hyperlinking, and the second
set lets package authors define new object names, with hyperlinking.  The
result is that references to Core SBML objects are displayed in black and
without linking, while package objects appear in blue, as hyperlinks.  The
linking is implemented using the standard \latex \texttt{hyperref} package.

In addition to these commands, \ninemlpkg provides other commands for
typesetting primitive type names (such as \primtype{SId}, \primtype{double},
and so on) and other entities and XML fragments.  They are described in
\vrefrange{primtype}{formatting-xml}.  The need for using these occurs
routinely when writing SBML specifications.

\subsubsection{Predefined SBML Core object and type names}
\label{predefined-classes}

\tab{sbmlcore} lists the commands to typeset the names of SBML Level~3 Core
objects.  They are designed to be as convenient to use as possible,
requiring only one additional character (i.e., the leading backslash
character) to be typed beyond the name of the object itself.

\begin{table}[htb]
  \rowcolors{2}{ninemlrowgray}{}
  \renewcommand{\arraystretch}{1.05}
  \begin{edtable}{tabular}{ll}
    \toprule
    \textbf{Command}                      & \textbf{Object} \\
    \midrule
    \cmd{AlgebraicRule}                   & \AlgebraicRule \\
    \cmd{Annotation}                      & \Annotation \\
    \cmd{AssignmentRule}                  & \AssignmentRule \\
    \cmd{Compartment}                     & \Compartment \\
    \cmd{Constraint}                      & \Constraint \\
    \cmd{Delay}                           & \Delay \\
    \cmd{EventAssignment}                 & \EventAssignment \\
    \cmd{Event}                           & \Event \\
    \cmd{FunctionDefinition}              & \FunctionDefinition \\
    \cmd{InitialAssignment}               & \InitialAssignment \\
    \cmd{KineticLaw}                      & \KineticLaw \\
    \cmd{ListOfCompartments}              & \ListOfCompartments \\
    \cmd{ListOfConstraints}               & \ListOfConstraints \\
    \cmd{ListOfEventAssignments}          & \ListOfEventAssignments \\
    \cmd{ListOfEvents}                    & \ListOfEvents \\
    \cmd{ListOfFunctionDefinitions}       & \ListOfFunctionDefinitions \\
    \cmd{ListOfInitialAssignments}        & \ListOfInitialAssignments \\
    \cmd{ListOfLocalParameters}           & \ListOfLocalParameters \\
    \cmd{ListOfModifierSpeciesReferences} & \ListOfModifierSpeciesReferences \\
    \cmd{ListOfPackages}                  & \ListOfPackages \\
    \cmd{ListOfParameters}                & \ListOfParameters \\
    \cmd{ListOfReactions}                 & \ListOfReactions \\
    \cmd{ListOfRules}                     & \ListOfRules \\
    \cmd{ListOfSpeciesReferences}         & \ListOfSpeciesReferences \\
    \cmd{ListOfSpecies}                   & \ListOfSpecies \\
    \cmd{ListOfUnitDefinitions}           & \ListOfUnitDefinitions \\
    \cmd{ListOfUnits}                     & \ListOfUnits \\
    \cmd{LocalParameter}                  & \LocalParameter \\
    \cmd{Message}                         & \Message \\
    \cmd{Model}                           & \Model \\
    \cmd{ModifierSpeciesReference}        & \ModifierSpeciesReference \\
    \cmd{Notes}                           & \Notes \\
    \cmd{Package}                         & \Package \\
    \cmd{Parameter}                       & \Parameter \\
    \cmd{Priority}                        & \Priority \\
    \cmd{RateRule}                        & \RateRule \\
    \cmd{Reaction}                        & \Reaction \\
    \cmd{Rule}                            & \Rule \\
    \cmd{SBML}                            & \SBML \\
    \cmd{SBase}                           & \SBase \\
    \cmd{SimpleSpeciesReference}          & \SimpleSpeciesReference \\
    \cmd{SpeciesReference}                & \SpeciesReference \\
    \cmd{Species}                         & \Species \\
    \cmd{StoichiometryMath}               & \StoichiometryMath \\
    \cmd{Trigger}                         & \Trigger \\
    \cmd{UnitDefinition}                  & \UnitDefinition \\
    \cmd{Unit}                            & \Unit \\
    \bottomrule
  \end{edtable}
  \caption{Commands for the names of object classes defined in the SBML Level~3
    Core specification.} 
  \label{sbmlcore}
\end{table}

\subsubsection{Defining new object names and types in package specifications}
\label{defining-classes}

When an SBML package specification document defines new object classes, it is
useful to make all mentions of the class name be hyperlinks to the class
definition in the document.  To this end, \ninemlpkg provides commands that can
be used to create new \latex commands to define hyperlinked name references.
The purpose of these commands is to let package authors define commands of
the form \cmd{\emph{ObjectName}} that print the name of the class and
simultaneously make it a hyperlink to the sections in the document where
\class{ObjectName} is defined.  These commands are best used conjunction with
\latex's \cmd{newcommand} command, to define custom macros in your document.

The first two commands in this category are \cmd{defRef} and \cmd{absDefRef}:

\begin{description}[font=\normalfont\ttfamily,style=nextline]

\item[\color{black}\textbackslash defRef\{\emph{name}\}\{\emph{section}\}]
  Create a hyperlinked reference to the section labeled \emph{section} and
  call it \emph{name}.  The reference is inserted at the point in the text
  where the \texttt{\textbackslash defRef} command is invoked.  This command
  is intended for references to regular (not abstract) classes; see the next
  command for abstract classes.

\item[\color{black}\textbackslash absDefRef\{\emph{name}\}\{\emph{section}\}]
  Create a hyperlinked reference to the section labeled \emph{section} and
  call it \emph{name}.  The reference is inserted at the point in the text
  where the \texttt{\textbackslash defRef} command is invoked.  This command
  is intended for references to abstract classes; see the previous command
  for the corresponding command for non-abstract classes.

\end{description}

The following is an example of how the commands above may be used; this is
taken straight from the source files of the SBML Level~3 Version~1 Core
specification document:

\begin{example}[style=latex]
\newcommand{\SBase}{\absDefRef{SBase}{sec:sbase}\xspace}
\newcommand{\SBML} {\defRef{SBML}{sec:sbml}\xspace}
\newcommand{\Model}{\defRef{Model}{sec:model}\xspace}
\end{example}

In these particular cases, the section labels ``\texttt{sec:base}'',
``\texttt{sec:sbml}'', etc., are defined using \latex's \cmd{label} command
at the beginning of each section of the specification document where the
corresponding objects are defined.  (In other words, everywhere a
``\cmd{section}'' or ``\cmd{subsection}'' or similar command is used in the
SBML Level~3 Version~1 Core document, it is followed with a ``\cmd{label}''.)
The command \cmd{xspace} is discussed in \sec{included-classes}.

Sometimes \notice it is desirable to write the names of object classes
\emph{without} introducing hyperlinks.  A common situation is when mentioning
the names of the object classes or types in section headings.  In these
situations, instead of using the \cmd{ObjectName} commands, you may use the
following commands provided by \ninemlpkg:

\begin{description}[font=\normalfont\ttfamily,style=nextline]

\item[\cmd{class\{\emph{name}\}}] Typesets \emph{name} in the same font style
  as used by \texttt{\textbackslash
    defRef\{}\emph{name}\texttt{\}\{}\emph{section}\texttt{\}}, without a
  hyperlink.

\item[\cmd{abstractclass\{\emph{name}\}}] Typesets \emph{name} in the same
  font style as used by \texttt{\textbackslash
    absDefRef\{}\emph{name}\texttt{\}\{}\emph{section}\texttt{\}}, without a
  hyperlink.

\end{description}



\subsubsection{Commands for formatting the names of primitive data types}
\label{primtype}

A convention that has evolved over the years of writing SBML specifications
is to typeset the names of primitive data types (such as \primtype{SId}) in a
monospaced, typewriter-like type face, without hyperlinks.  SBML packages may
define their own new primitive types.  To format the names of these types in
a style consistent with the SBML Level~3 Core specification document, package
authors should use the \cmd{primtype} and \cmd{primtypeNC} commands:

\begin{description}[font=\normalfont\ttfamily,style=nextline]

\item[\cmd{primtype\{\emph{name}\}}] Typesets the name of a primitive data
  type \emph{name}.  SBML Level~3 Core also defines and uses a number of
  primitive data types, but these do not have separate commands in \ninemlpkg.
  Instead, they should be written using the command
  \mbox{\cmd{primtype\{\emph{type}\}}}, where \emph{type} is one of the
  following names:

  \begin{center}\ttfamily\color{black}
    \begin{edtable}{tabular}{llllll}
      boolean	& ID	& positiveInteger & SId  	& string	& UnitSIdRef\\
      double	& int	& SBOTerm	  & SIdRef	& UnitSId \\
    \end{edtable}
  \end{center}

\item[\cmd{primtypeNC\{\emph{name}\}}] Like \cmd{primtype}, but does not
  force the color of the text to be black.  The main use of this variant of
  the command is when writing the names of primitive types in the arguments
  to \latex sectioning commands (e.g., \cmd{section}, \cmd{subsection} and
  the like), to avoid the change in color that would otherwise occur in the
  document's table of contents.  (The \cmd{primtype} command sets the color
  of the text to pure black, to make the text stand out more in the document
  body.  The change of color would occur because the entries in the table of
  contents are all hyperlinks to the beginning of the sections, and
  hyperlinks are colored blue.)

\end{description}

For example, the sequence ``\cmd{primtype\{SId\}}'' written somewhere in the
body of a document will produce ``\primtype{SId}'' in the formatted output;
by contrast, the following illustrates how to write a type name in a section
heading:

\begin{example}[style=latex]
\subsection{This is the documentation for \primtypeNC{MySpecialSId}}
\end{example}

A \warning small spacing problem can occur when \cmd{primtypeNC} is used in
\cmd{subsubsection} and \cmd{paragraph} titles: the content of the
\cmd{primtypeNC} may end up too close to the preceding text in the section
title.  The problem occurs because the typeface used for \cmd{subsubsection}
and \cmd{paragraph} is slanted whereas the fixed-width typeface of
\cmd{primtypeNC} is not.  To prevent the unattractive compressed-looking
result, add a small amount of space (usually 1 point is sufficient) using
\cmd{hspace}, like so:

\begin{example}[style=latex]
\subsubsection{This is the documentation for \hspace*{1pt}\primtypeNC{SomeOtherId}}
\end{example}

The above is typically not necessary in \cmd{section} or \cmd{subsection},
nor when the invocation of \cmd{primtypeNC} occurs as the first thing in a
\cmd{subsubsection}.


\subsubsection{Commands for formatting in-text XML descriptions}
\label{formatting-xml}

In addition to formatting the names of SBML object classes and primitive
data types, \ninemlpkg provides commands for formatting text meant to be
literal examples of XML.  These special commands are provided so in-text
descriptions of XML constructs can be formatted in a way and attractive
way:

\begin{description}[font=\normalfont\ttfamily\color{black},style=nextline]

\item[\cmd{token\{\emph{text}\}}] Formats literal XML tokens, such as
  attribute names.  Do not use this to format text with embedded spaces;
  instead, use multiple \texttt{\textbackslash token} commands separated by
  spaces.

\item[\cmd{tokenNC\{\emph{text}\}}] Like \texttt{\textbackslash token}, but
  does not force the color of the text to be black.  The main use of this
  variant of the command is when writing the names of primitive types in the
  arguments to \latex sectioning commands (e.g., \cmd{section} and the like),
  to avoid the change in color that would otherwise occur in the document's
  table of contents.

\item[\cmd{val\{\emph{text}\}}] Format the value of an attribute.  This is
  essentially \cmd{token} but surrounded by double quotes.

\item[\cmd{uri\{\emph{text}\}}] Format a URI.  This is essentially \cmd{val};
  it's provided make the formatting of URIs in documents more consistent.

\end{description}

The command \cmd{token} and others above are only meant for single in-text
mentions of tokens and XML attribute-value pairs.  They are not suitable
for longer content; for those cases, use the \texttt{example} environment
(see \sec*{example-env}), which is intended for multiline literal content.


\subsection{Line numbers}
\label{about-lineno}

\ninemlpkg preloads the \latex package \texttt{lineno} and configures it to
produce the line numbers in the right column of every page.  The numbers are
important for specification documents because they allow discussion and bug
reports to refer to specific portions of the text.  Crucially, lists of
issues recognized \emph{after} a specification is released need ways of
referring to precise locations in the document, and line numbers are
invaluable for that purpose.

For \notice the most part, you do not need to do anything in your document to
get line numbers to appear.  There are exceptions: certain content such as
tabular material and floats are not handled by \texttt{lineno} very well, and
require manual intervention.  Some cases cannot be fixed at all, notably
figures incorporated from external files, but tabular material is fixable.
To get line numbers to be displayed in tables, wrap all uses of
\texttt{tabular} with the special environment \texttt{edtable}.  The basic
idiom is the following:

\begin{example}[style=latex]
\begin{edtable}{tabular}{|\color{gray}\emph{...normal tabular column specifiers...}|}
  |\color{gray}\emph{...tabular content...}|
\end{edtable}
\end{example}

The \texttt{edtable} environment is able to wrap a number of standard
\latex environments, of which \texttt{tabular} is probably the most useful
for most kinds of tables.  Practical examples of using \texttt{edtable}
appear elsewhere in this document.


\subsection{SBML validation rules}
\label{validation-rules}

A convention developed for the main SBML specification documents is to define
validation and consistency rules that must or should be satisfied by
documents that conform to the specification.  SBML package specifications
should likewise define their own validation and consistency rules.

The SBML convention identifies different degrees of strictness.  The
differences are expressed in the statement of a rule: either a rule states a
condition \emph{must} be true, or a rule states that it \emph{should} be
true.  Rules of the former kind are SBML validation rules---a model encoded
in SBML must conform to all of them in order to be considered valid.  Rules
of the latter kind are consistency rules.  To help highlight these
differences, the SBML specification documents and \ninemlpkg provide commands
to format the three kinds of rules, with three different symbols:

\begin{description}

\item[\hspace*{6.5pt}\vSymbol\vsp] A \vSymbolName indicates a
  \emph{requirement} for conformance. If a model fails to follow this rule,
  it does not conform to the specification.  (Mnemonic intention behind the
  choice of symbol: ``This must be checked.'')

\item[\hspace*{6.5pt}\cSymbol\csp] A \cSymbolName indicates a
  \emph{recommendation} for model consistency.  If a model does not follow
  this rule, it is not considered strictly invalid as far as the
  specification is concerned; however, it indicates that the model contains
  a physical or conceptual inconsistency.  (Mnemonic intention behind the
  choice of symbol: ``This is a cause for warning.'')

\item[\hspace*{6.5pt}\mSymbol\msp] A \mSymbolName indicates a strong
  recommendation for good modeling practice.  This rule is not
  strictly a matter of SBML encoding, but the recommendation comes
  from logical reasoning.  As in the previous case, if a model does
  not follow this rule, it is not strictly considered an invalid SBML
  encoding.  (Mnemonic intention behind the choice of symbol: ``You're
  a star if you heed this.'')

\end{description}

\ninemlpkg defines three commands for writing these rules in SBML package
specifications documents:

\begin{description}[font=\normalfont\ttfamily,style=nextline]

\item[\cmd{validRule\{\emph{number}\}\{\emph{text}\}}] Format \emph{number}
  as a validation rule with the description \emph{text}.

\item[\cmd{consistencyRule\{\emph{number}\}\{\emph{text}\}}] Format \emph{number}
  as a consistency rule with the description \emph{text}.

\item[\cmd{modelingRule\{\emph{number}\}\{\emph{text}\}}] Format \emph{number}
  as a modeling rule with the description \emph{text}.

\end{description}

SBML specifications typically gather all such rules into an appendix at the
end of the document.


\subsection{Document flags and notes}
\label{document-notes}

Sometimes it is useful to flag content in a document to draw readers'
attention to it.  It is also sometimes useful during the development of a
document to be able to leave coments for coauthors and readers.  \ninemlpkg
provides a few commands for these purposes.

\begin{description}[font=\normalfont\ttfamily,style=nextline]

\item[\cmd{notice}] Puts \notice a hand pointer in the left margin
  (illustrated at the left).  The symbol will always be displayed.

\item[\cmd{warning}] Puts \warning a warning sign in the left margin
  (illustrated at the left).  The symbol will always be displayed.

\item[\cmd{draftnote\{\emph{text}\}}] Writes \draftnoteInternal{The note
    text will only appear for documents in draft mode.} the \emph{text} as a
  yellow margin note (illustrated at the left), but only if the document
  flag \texttt{[draftspec]} is given to the \cmd{documentclass} command.
  If the flag is not given, nothing is displayed in the margin.

\end{description}

The commands \cmd{notice} and \cmd{warning} are intended to be used for
things that should be left in all versions of a specification document,
whether draft or final.  The command \cmd{draftnote}, on the other hand, is
for content that is only meant to be included in draft versions of the
document.


% -----------------------------------------------------------------------------
% \section{Additional conventions for SBML specification documents}
% \label{conventions}
% -----------------------------------------------------------------------------



% -----------------------------------------------------------------------------
% \section{What \ninemlpkg provides: a concise reference}
\section{Additional features of \ninemlpkghead}
\label{provides}
% -----------------------------------------------------------------------------

% This section describes the commands and features provided by \ninemlpkg.

This section describes some additional aspects of \ninemlpkg.


\subsection{Options understood by \ninemlpkghead}
\label{pkg-options}

A number of options may be given to \ninemlpkg in the \cmd{documentclass}
command that begins a document.  Here follows a complete list:

\begin{description}[font=\normalfont\ttfamily\color{black},style=nextline]

\item[draftspec] This option causes the front page of the document to contain
  the word ``DRAFT'' in large gray letters, and the footer of every page of
  the document to contain the word ``(DRAFT)''.  Authors should use this
  option until such time as the specification document is considered a
  release candidate or a final release.

\item[finalspec] (Default) This option causes the large ``DRAFT'' on the
  front page and ``(DRAFT)'' in the footers to be omitted.  It is the
  opposite of the \texttt{draftspec} option.

\item[toc] (Default) This option causes \ninemlpkg to include a table of
  contents as the second page of the document.  Whether the table has one
  wide column or two narrow columns is then controlled by the options
  \texttt{twocolumntoc} and \texttt{singlecolumntoc}, described below.

\item[notoc] This option causes the table of contents to be omitted.  (It
  is unclear under what circumstances one would want to omit the table of
  contents, but since some other \latex classes include this feature,
  \ninemlpkg follows suit.)

\item[twocolumntoc] This option causes \ninemlpkg to produce a two-column table
  of contents rather than the default one-column version.  (That is, unless
  the \texttt{notoc} option is also given, in which case, no table of
  contents is produced.)  This is useful when a document is long, with many
  sections, because the two-column version is more compact.  (However, beware
  that since the columns are narrower than the single-column version, long
  section names may become wrapped, leading to an aesthetically less pleasing
  result.  If you need to use two column output, you may alway want to
  examine whether you can shorten your section titles.)

\item[singlecolumntoc] (Default) This option is the opposite of
  \texttt{twocolumntoc}; it causes \ninemlpkg to produce a single column table
  of contents as the second page of the document.  (Again, if \texttt{notoc}
  is also given, then no table of contents is produced at all.)

\end{description}



% \subsection{Commands defined by \ninemlpkg}
% \label{cmds-defined}

% notice
% address
% package* commands


\subsection{Notable \latex packages preloaded by \ninemlpkghead}
\label{included-classes}

As mentioned above, \ninemlpkg preloads many common \latex packages.  Some are
used to implement the features of \ninemlpkg itself; others are preloaded so
that authors do not have to load them explicitly.  Knowing what \ninemlpkg
provides upfront also makes it easier to explain how to write SBML
specification documents.

The following is a list of the \latex packages preloaded by \ninemlpkg.  It is
beyond the scope of this document to explain their features and capabilities
in detail; readers are urged to consult the documentation for each one to
learn more about them.  Documentation is available at CTAN
(\url{http://www.tug.org/ctan.html}).


\begin{itemize}

\item \texttt{amssymb}: This package defines many symbols and special
  characters.  In \ninemlpkg, it is used to get the symbols defined by the
  validation rule commands \cmd{validRule}, \cmd{consistencyRule} and
  \cmd{modelingRule} described in \sec{validation-rules}.

\clearpage

\item \texttt{amsmath}: This package defines many additional symbols and
  macros for mathematics.  It is not actually used in \ninemlpkg, but it is
  popular, and to make it work properly in combination with \ninemlpkg, some
  corrections to delimiter sizes need to be introduced.  Therefore,
  \texttt{amsmath} is included too.

\item \texttt{array}: This provides a new and extended implementation of
  the \latex \texttt{array} environment and \texttt{tabular}
  environments.  It is particularly useful for the column formatting
  options it provides for the \texttt{tabular} environment. 

\item \texttt{bbding}: A font set that provides a variety of symbol glyphs.

\item \texttt{booktabs}: Mentioned in \sec*{about-tables}, this \latex
  package defines the commands \cmd{toprule}, \cmd{bottomrule}, and
  \cmd{midrule} for produced more attractive and professional-looking
  tables.

\item \texttt{enumitem}: This relatively new package adds facilities for more
  easily adjusting the look of \latex list environments, including the
  \texttt{description} environment---something that is difficult to do in
  plain \latex.  (Oh sure, there \emph{are} variables and commands for doing
  it in \latex, but they are limited and sometimes produce unexpected
  consequences elsewhere in a document.)

\item \texttt{graphicx}: A powerful and rich system for working with
  external graphics files in \latex documents. 

\item \texttt{hyperref}: Mentioned in \sec{hyperlinks}, this package defines
  commands for creating hyperlinks within and between documents.  \ninemlpkg
  uses it to define commands such as \cmd{fig} and \cmd{tab}.

\item \texttt{lineno}: Used by \ninemlpkg to implement line numbers on the
  page.  As mentioned in \sec{about-lineno}, not all line numbering can be
  accomplished completely without user intervention; in some cases, authors
  must add special commands to get line numbers into content on the page.

\item \texttt{listings}: Used by \ninemlpkg to implement the \texttt{example}
  environment and \cmd{exampleFile} command.

\item \texttt{multicol}: This provides an environment for putting multiple
  columns of text on a page.  In \ninemlpkg, it is used to produce two-column
  table of contents when the \texttt{[twocolumntoc]} option is given as
  described in \sec{pkg-options}.

\item \texttt{natbib}: This reimplements the normal \latex \cmd{cite} command
  to work with author-year citations and add a number of useful features.  It
  provides variants of \cmd{cite} such as \cmd{citep} and \cmd{citeauthor}.
  The features of \texttt{natbib} are worth learning and using.

\item \texttt{overpic}: The \texttt{overpic} package provides the ability
  to write \latex content on top of a figure.  This is particularly handy
  when including graphics in PDF format and you need to insert, for
  example, section references or other information that is generated
  dynamically.

\item \texttt{rotating}: The \texttt{rotating} package provides commands for
  rotating figures on a page.  This is particularly useful for rotating large
  tables on a page by 90\textdegree.

\item \texttt{varioref}: This defines commands such as \cmd{vref}, which is
  similar to \cmd{ref} but includes a page reference such as ``on the next
  page'' or ``on page 27'' when the corresponding \cmd{label} is not on the
  same page.  (The way that the page references are generated requires
  multiple passes of running \latex, which is one of the reasons why using
  \ninemlpkg requires 2--3 runs of \latex to generate final cross-references.)

\item \texttt{varwidth}: This defines a \texttt{varwidth} environment that
  acts much like \texttt{minipage}, but produces output with a width that is
  the natural width of its contents.

\item \texttt{wasysym}: Similar to \texttt{amssymb}, this \latex package
  defines another set of symbols and special characters.  

\item \texttt{xcolor}: This package defines a large number of color names and
  associated commands.

\item \texttt{xspace}: A marvelous little extension that provides just one
  very useful command, \cmd{xspace}.  The \cmd{xspace} command can be used
  at the end of a macro designed to be expanded into text; it adds a space
  unless the macro is followed by a punctuation symbol.  This makes it
  possible to invoke the macro in text without adding an empty
  \texttt{\{\}} pair after it---something that would otherwise be a problem
  because \latex would consume the space character following the macro
  invocation, leading to a missing space in the final output.  The commands
  such as \cmd{SBML} are defined using \cmd{xspace}, so that in running
  text, you can write ``one two \cmd{SBML} three four'' and it will produce
  ``one two \SBML three four'' rather than ``one two \class{SBML}three
  four''.

\end{itemize}



% \subsection{Other potentially useful \latex classes \emph{not} preloaded by \ninemlpkg}

% There are \emph{many} \latex classes available in the world.  It would be
% impossible to do them all justice here, but authors may find some of the
% following useful in the current context of SBML specification documents.

% tikz
% mhchem



\section{Acknowledgments}

This work was made possible by grant R01 GM070923 from the NIH National
Institute of General Medical Sciences (USA) for continued development and
support of SBML and related software infrastructure.


\clearpage
\bibliography{sbmlpkgspec}



% -----------------------------------------------------------------------------
% End of document
% -----------------------------------------------------------------------------

\end{document}
